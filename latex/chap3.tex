\chapter{Framing}
\label{chap:framing}

\begin{figure}
\centering
\begin{forest}
[,circle,draw, for tree={grow=east,anchor=west,child anchor=west}
	[{\hyperref[apartheid_frame]{\emph{Apartheid} cluster} (Figure~\ref{apartheid_frame})}]
	[67
		[{\hyperref[arrastao_frame]{\emph{Arrastão} cluster} (Figure~\ref{arrastao_frame})}]
		[{\hyperref[middleground_frame]{\emph{Middle ground} cluster} (Figure~\ref{middleground_frame})}]
	]
]
\end{forest}
\caption{Root of the tree and the three main frames.}
\label{cluster_analysis}
\end{figure}

\begin{figure}
\centering
\resizebox{!}{\dimexpr\textheight-2cm}{
\begin{forest}
[68, for tree={grow=east,anchor=west,child anchor=west}
	[85
		[89
			[\autocite{youtube_pcsiqueira}]
			[99 [\autocite{diegoquinteiro}] [\autocite{documentario_2000}]]
		]
		[77
			[91
				[\autocite{maria_fro}]
				[90
					[\autocite{g1_naspalavras}]
					[93
						[88 [\autocite{folha_laura}] [\autocite{elpais_apartheid}]]
						[86 [\autocite{rosana}] [\autocite{folhapolitica_medo_brancos}]]
					]
				]
			]
			[70
				[93
					[\autocite{boff_rolezinhos}]
					[87 [\autocite{brum_vandalos}] [\autocite{sakamoto_rolezinho}]]
				]
				[74 [\autocite{brum_kaique}] [\autocite{terra_rolezeiros_ujs}]]
			]
		]
	]
	[83
		[98
			[\autocite{cartacapital_sheherazade}]
			[70
				[\autocite{piaui_rolezinho_miami}]
				[71 [\autocite{pha}] [\autocite{folhapolitica_rolezinho_congresso}]]
			]
		]
		[95
			[83
				[\autocite{veto_consagra_apartheid}]
				[99 [\autocite{blogdacidadania}] [\autocite{folha_direito_segregacao}]]
			]
			[91
				[86
					[\autocite{neder}]
					[95 [\autocite{folha_pms_agredindo}] [\autocite{folha_vanessa}]]
				]
				[92
					[89
						[87
							[\autocite{hitler_rolezinhos}]
							[86 [\autocite{uol_grifes}] [\autocite{folha_elite_classec}]]
						]
						[94
							[\autocite{folha_carvalho}]
							[87
								[\autocite{folha_medo_brancos}]
								[93 [\autocite{folha_jovens_fas}] [\autocite{elpais_rebeliao}]]
							]
						]
					]
					[97
						[94 [\autocite{youtube_fea_usp}] [\autocite{folha_rolezinho_USP}]]
						[86 [\autocite{brasildefato_pochman}] [\autocite{gazeta_digital}]]
					]
				]
			]
		]
	]
]
\end{forest}
}
\caption{The \emph{Apartheid} cluster. Internal nodes are percentages that estimate how closely related are the two subtrees under that internal node. The leaves are the articles. There is a collapsed and annotated version of this tree in Figure~\ref{apartheid_frame_annotated}.}
\label{apartheid_frame}
\end{figure}

\begin{figure}
\centering
\begin{forest}
[72, for tree={grow=east,anchor=west,child anchor=west}
	[80
		[78
			[\autocite{reinaldo_mistificacoes}]
			[78 [\autocite{reinaldo_advogado}] [\autocite{folha_ponde}] ]
		]
		[78
			[100 [\autocite{reinaldo_datafolha}] [\autocite{reinaldo_paucome}] ]
			[75 [\autocite{constantino_barbarie}] [\autocite{folha_datafolha}] ]
		]
	]
	[74
		[80
			[\autocite{jornaldejundiai}]
			[94 [\autocite{uol_rolezeiras}] [\autocite{folha_shoppings_rolezodromos}] ]
		]
		[93
			[94
				[\autocite{folha_onda}]
				[82
					[83 [\autocite{bandnewstv_arrastao_itaquera}] [\autocite{folha_arrastao_itaquera}] ]
					[92
						[\autocite{folha_liminar_shoppings}]
						[94
							[\autocite{g1_guardaespancado}]
							[100 [\autocite{folha_confronto_itaquera}] [\autocite{g1_guarulhos}] ]
						]
					]
				]
			]
			[70
				[\autocite{youtube_rachel}]
				[81
					[99 [\autocite{youtube_primeiro_rolezinho_outside}] [\autocite{youtube_parodia_rolezeiras}] ]
					[93 [\autocite{youtube_away}] [\autocite{diva_rolezeiras}] ]
				]
			]
		]
	]
]
\end{forest}
\caption{The \emph{arrastão} cluster. Internal nodes are percentages that estimate how closely related are the two subtrees under that internal node. The leaves are the articles. There is a collapsed and annotated version of this tree in Figure~\ref{arrastao_frame_annotated}.}
\label{arrastao_frame}
\end{figure}

\begin{figure}
\centering
\begin{forest}
[98, for tree={grow=east,anchor=west,child anchor=west}
	[\autocite*{youtube_caue_moura}]
	[98 [\autocite{beguoci_rolezinhos}] [\autocite{revista_bula}]]
]
\end{forest}
\caption{The \emph{middle ground} cluster. Internal nodes are percentages that estimate how closely related are the two subtrees under that internal node. The leaves are the articles. There is an annotated version of this tree in Figure~\ref{middleground_frame_annotated}.}
\label{middleground_frame}
\end{figure}

The result of the hierarchical cluster analysis and subsequent frame validation---described in the \nameref{sec:methods} section---is presented in Figure~\ref{cluster_analysis}. Three subtrees were identified as three distinct frames and named after keywords or concepts associated to them: \emph{apartheid} (Figure~\ref{apartheid_frame}), \emph{arrastão} (Figure~\ref{arrastao_frame}), and \emph{middle ground} (Figure~\ref{middleground_frame}).

Those three frames represent three different stands on \emph{rolezinhos}. The \emph{arrastão} frame tends to criminalize the gatherings and criticize the youth. On the other side of the ideological spectrum, the \emph{apartheid} frame supports the \emph{rolezinhos} and finds the source of the problem elsewhere: society, government, malls, police, and other institutional forces. Finally, the \emph{middle ground} frame criticizes both conservatives and progressives for appropriating the controversy and using it in narrow-minded cultural wars. 

The following sections in this chapter will unravel the inner structure of the binary trees that correspond to each frame. Before starting, it is worth providing a comment about the organization of this chapter.

Two previous projects based on Media Cloud---the aforementioned studies on SOPA/PIPA \autocite{benkler2013social} and Trayvon Martin \autocite{graeff2014battle} controversies---opted for presenting the evolution of frames in a single timeline. However, there is a significant difference between those issues and the controversy on the \emph{rolezinhos} in Brazil: at the end of the day, SOPA/PIPA was rejected by Congress and the Trayvon Martin case engendered a debate on race, rather than on other possible topics, like gun control or violence. In both cases, it was possible to identify winning and losing frames.

In the case of \emph{rolezinhos}, this distinction is less clear because it is not possible to identify a clear conclusion for the controversy. The \emph{rolezinhos} gradually disappeared from the media as other topics gained prominence in the public agenda.

Therefore, in this chapter, we preferred to individually address the temporal evolution of each frame without bringing them together in the same timeline. The intention was to avoid the impression of an epilogue for the controversy, something that is not supported by the data.

\section{Arrastão}

According to 37\% of the texts in our 60-article sample, \emph{rolezinhos} are unlawful. Those articles may differ on the best way to deal with the youth or if robberies and drug trafficking did occur during the events, but they agree on one essential point: at least for the sake of safety and civility, those gatherings should not happen in malls. Those articles constitute the \emph{arrastão frame} (Figure~\ref{arrastao_frame_annotated}).

\begin{figure}
\centering
\begin{forest}
[72, for tree={grow=east,anchor=west,child anchor=west}
	[{\hyperref[neocons_subcluster]{\emph{Neocons} subcluster} (Figure~\ref{neocons_subcluster})}]
	[74
		[80
			[{\autocite[][\emph{Rolezinho} in a neighboring town]{jornaldejundiai}}]
			[94
				[{\autocite[][Video: Girls in \emph{rolezinhos}]{uol_rolezeiras}}]
				[{\autocite[][\emph{Folha de S. Paulo}: The malls' perspective]{folha_shoppings_rolezodromos}}]
			]
		]
		[93
			[{\hyperref[traditionalmedia_subcluster]{\emph{Traditional media} subcluster} (Figure~\ref{traditionalmedia_subcluster})}]
			[{\hyperref[youtube_subcluster]{\emph{YouTube} subcluster} (Figure~\ref{youtube_subcluster})}]
		]
	]
]
\end{forest}
\caption{The \emph{arrastão} cluster. A collapsed and annotated version of Figure~\ref{arrastao_frame}.}
\label{arrastao_frame_annotated}
\end{figure}

On December \nth{7}, some people present at the first \emph{rolezinho}, especially customers at the mall, posted their impressions on Twitter \autocite{twitter_arrastao_itaquera} and Internet forums \autocite{forum_jogos_uol}. Many of them used the term \emph{arrastão} to describe what they had seen.

\emph{Arrastão} is a tactic of collective theft in which a group of people invades a crowded space and demands money, jewelry, and sometimes even clothes or shoes of the disoriented passersby \autocite{npr_arrastao}. Perplexed by the screams and the hustle and bustle, the victims become easy prey for the thieves. The term began to be used in the 80s, but became famous in the early 90s, when an \emph{arrastão} at Ipanema beach made the headlines worldwide \autocite{youtube_arrastao}.

Since its origins, the term \emph{arrastão} has a hint of social and racial discrimination \autocite{francisco2003arrastao}. It became part of the Brazilian sociological lexicon and is often used, even when there is no evidence of criminal activity, as an alert when a contingent of black and/or poor youth abruptly occupies a public area, such as a beach, a mall, a park, or an avenue. A cry of \emph{arrastão} in any of those environments usually cause a fear reaction tantamount to the announcement of \enquote{fire} in a crowded theater.

\subsection*{Traditional media}

In the course of the controversy, the term \emph{arrastão} gradually lost importance to the point of almost disappearing, as shown in Figure~\ref{chart_arrastao}.

\begin{figure}
\begin{center}
\begin{asy}
real[] percentageOfArrastao = {84.62, 66.67, 47.22, 50.00, 18.18, 22.06, 5.42, 3.14, 2.09, 1.81, 1.24, 0.00, 0.00};
combo_chart("\% \emph{Arrastão}", percentageOfArrastao);
\end{asy}
\end{center}
\caption{Weekly percentage of stories that mention the word \emph{arrastão}.\label{chart_arrastao}}
\end{figure}

Interestingly, the headline about the first \emph{rolezinho} in \emph{Folha de S. Paulo}---Brazil's main newspaper---changed two days after being published to eliminate the term \emph{arrastão}. A search in the Wayback Machine at the Internet Archive shows that, on December \nth{8}, the headline was \blockcquote{wayback_arrastao}[.]{Teens make an \emph{arrastão} and mall closes earlier in SP} Two days later, the word \emph{arrastão} disappeared from the title replaced by \blockcquote{folha_arrastao_itaquera}[.]{Teens schedule a meeting through internet and cause turmoil in a SP mall}

Something similar happened at \emph{Folha de S. Paulo}'s main competitor \emph{O Estado de S. Paulo}. In the subsequent week, the paper published an article with the title \blockcquote{diariocomercio_arrastao}[.]{Teens make an \emph{arrastão} in a shopping mall in Guarulhos} A few hours later, the headline changed to \blockcquote{estado_guarulhos}[.]{Mess in a mall in Guarulhos ends up with 23 detained}

Very soon the majority of the media outlets realized that \emph{arrastão} was not the most appropriate word to define a \emph{rolezinho}. However, traditional media continued to have a conservative first reaction to the gatherings. Notwithstanding the dispassionate informative tone, the first texts often portrayed the events as harmful, messy, or dangerous. So much so that the first articles published immediately after the \emph{rolezinhos} constitute a well-defined subcluster inside the \emph{arrastão} frame as shown in Figure~\ref{traditionalmedia_subcluster}.

\begin{figure}
\centering
\begin{forest}
[94, for tree={grow=east,anchor=west,child anchor=west}
	[{\autocite[][\emph{Folha de S. Paulo}: Wave of \emph{rolezinhos}]{folha_onda}}]
	[82
		[83
			[{\autocite[][\emph{Band}: First \emph{rolezinho}]{bandnewstv_arrastao_itaquera}}]
			[{\autocite[][\emph{Folha de S. Paulo}: First \emph{rolezinho}]{folha_arrastao_itaquera}}]
		]
		[92
			[{\autocite[][\emph{Folha de S. Paulo}: Court injunctions against \emph{rolezinhos}]{folha_liminar_shoppings}}]
			[94
				[{\autocite[][\emph{G1}: Security guard is beaten in \emph{rolezinho}]{g1_guardaespancado}}]
				[100
					[{\autocite[][\emph{Folha de S. Paulo}: Rubber bullets]{folha_confronto_itaquera}}]
					[{\autocite[][\emph{G1}: Second \emph{rolezinho}]{g1_guarulhos}}]
				]
			]
		]
	]
]
\end{forest}
\caption{The \emph{traditional media} subcluster.}
\label{traditionalmedia_subcluster}
\end{figure}

Interestingly, that bias seems a knee-jerk feature of the first article after a \emph{rolezinho}. Subsequent stories usually had more nuanced statements and richer contextualization. Op-eds were also more diverse in their interpretations. Taking into account that \emph{rolezinhos} happened over the weekends in the late afternoon or at night, it is possible to venture an explanation. Those articles were usually written by lone reporters, on duty during the weekend. They were not able to visit the mall, so they had to rely on police reports---usually a staunch conservative source---by telephone.

In fact, a note published in the corrections section of \emph{Folha de S. Paulo} on December \nth{9}---after the first \emph{rolezinho}---supports this hypothesis \autocite{folha_erramos_arrastao_itaquera}. The text attributes to the police the wrong information that there was an \emph{arrastão} in \emph{Lojas Americanas} (a popular chain store) during the \emph{rolezinho} in \emph{Shopping Metrô Itaquera}. In fact, not only there was no \emph{arrastão} but also there is no \emph{Lojas Americanas} in \emph{Shopping Metrô Itaquera}.

First impressions are frequently lasting impressions. Therefore, the framing influence of those first stories must not be underestimated. The most popular piece in this subcluster---a report of the first \emph{rolezinho} by Band \autocite{band_arrastao_itaquera}---had a \emph{total count} of 62,727, more than 3\% of the total media attention in the overall corpus. 

Another relevant piece for the \emph{arrastão} rhetoric was \emph{G1}'s \blockcquote{g1_guardaespancado}{Guard is assaulted during \enquote{rolezinho} in Guarulhos} on January, \nth{13} (total count: 11,236). That text subverted the traditional \emph{apartheid} frame---the policeman was the victim and the youth was the aggressor. The participants in the \emph{rolezinho} were allegedly drinking alcohol in a public park and, when the guard rebuked them, they stoned him.

\emph{G1}'s article---and the related video---went viral and became a frequent reference among supporters of the \emph{arrastão frame}. Conservative commentator Rodrigo Constantino, for instance, recommended that video to \blockcquote{constantino_comportamental}[.]{those who think that every criticism of \emph{rolezinhos} comes from racial or social prejudice}

\subsection*{YouTube}
\label{sub:youtube}

In the first week of the controversy, the main group that spoke out against the use of the term \emph{arrastão} were the shopkeepers and shopping center owners, both individually and through their associations. In the very first article about the first \emph{rolezinho} \autocite{band_arrastao_itaquera}, the administration of \emph{Shopping Metro Itaquera} denied an \emph{arrastão} and even any sort of theft inside the mall. That was the consensus amongst the shopping malls throughout the controversy as it is demonstrated by a qualitative analysis of the 496 unique sentences in the corpus that mention the Association of Shopping Center's Retailers (Alshop), its chairman Nabil Sayhoun, or the Brazilian Association of Shopping Centers (Abrasce). They wanted to emphasize that the malls continued to be the safest spaces for consumption and leisure.

Rachel Sheherazade, an influential conservative commentator, suggested that the malls would rather deny the occurrence of thefts in order to \enquote{overshadow the negative propaganda} of \emph{rolezinhos} because many customers would cease to go shopping on Christmas Eve if they believed in the risk of \emph{arrastão} in shopping malls \autocite{youtube_rachel}. Her commentary was aired on the TV station SBT on December 16th, right after the second \emph{rolezinho} at \emph{Shopping Internacional de Guarulhos}, when 23 young people were detained and released shortly thereafter. Sheherazade's video achieved great prominence in Facebook (total count: 45,339). She also supported a bold police action against the \emph{rolezinhos}. That piece became one of the high-ranking stories in defense of the frame \emph{arrastão}.

Sheherazade's comment belongs to the \emph{YouTube} subcluster (Figure~\ref{youtube_subcluster}), a group of articles that have two things in common: they are videos and they convey a negative---although very simple---message about the \emph{rolezinhos}. There is no room for elaborate sociological reflections.

\begin{figure}
\centering
\begin{forest}
[70, for tree={grow=east,anchor=west,child anchor=west}
	[{\autocite[][Sheherazade: \emph{Rolezinho} as \emph{arrastão}]{youtube_rachel}}]
	[81
		[99
			[{\autocite[][Chaotic footage: First \emph{rolezinho}]{youtube_primeiro_rolezinho_outside}}]
			[{\autocite[][Parody: Girls in \emph{rolezinho}]{youtube_parodia_rolezeiras}}]
		]
		[93
			[{\autocite[][Comedian Away Nilzer on \emph{rolezinhos}]{youtube_away}}]
			[{\autocite[][Comments on video \enquote{Girls in \emph{rolezinho}}]{diva_rolezeiras}}]
		]
	]
]
\end{forest}
\caption{The \emph{YouTube} subcluster.}
\label{youtube_subcluster}
\end{figure}

One of them does not convey any explicit message at all. It is a YouTube video recorded with a mobile phone in the first \emph{rolezinho} \autocite{youtube_primeiro_rolezinho_outside}. At the beginning, the images are so chaotic that it is difficult to make sense of it. Then, when the private guards arrive to disperse the crowd, they are harassed by some youth. The resulting impression about the participants is unfavorable.

The other YouTube videos are (tentatively) humorous. Comedian Away Nilzer, for instance, is largely nonsensical and seems intoxicated \autocite{youtube_away}. He menaces the viewer---imagined as a participant in a \emph{rolezinho}---with a knife while insulting him. When it is possible to understand him, it becomes plain that he is castigating the youth and their life style. He is also rude, even suggesting that the participants go to the mall only to have sexual intercourse in the restrooms. Nevertheless, he is the fifth most popular link in the corpus (total count: 51,534).

There are also two articles that make fun of girls who go to \emph{rolezinhos}. Both elaborate on a more serious video produced by \emph{UOL} that describes the behavior of those adolescents. That original piece belongs to the \emph{arrastão frame} too \autocite{uol_rolezeiras}, but it is more ambiguous: it gives room to criticize the girls for their frivolity but also to empathize with them for their naiveté and kindness.

The altered versions in the \emph{YouTube subcluster}, however, do not leave much room for interpretation. One of them presents short sequences of the original video accompanied by sarcastic comments about the girls' clothes and ideas \autocite{diva_rolezeiras}. The other is more hostile. The original speech was substituted by degrading remarks. The new lines convey three main ideas: those girls hate going to school, they are promiscuous, and participants in \emph{rolezinhos} are used to stealing from people \autocite{youtube_parodia_rolezeiras}.

Together, the five videos are remarkably influential. They represent 9.8\% of the social media attention in the overall corpus.

\subsection*{Neocons}
\label{sec:neocons}

Were it not for the need of being brief and simple on TV, aforesaid Rachel Sheherazade would probably be placed in another subcluster: the \emph{neocons}. She belongs to a new generation of communicators and intellectuals who are challenging the progressive establishment, especially in the most influential media outlets.

Due to their ideological cohesiveness, all of them---except Sheherazade---are gathered in the same subcluster (Figure~\ref{neocons_subcluster}). They usually mention the \emph{rolezinhos} issue in the context of the Brazilian cultural wars and to make a point against their left wing adversaries.

\begin{figure}
\centering
\begin{forest}
[80, for tree={grow=east,anchor=west,child anchor=west}
	[78
		[{\autocite[][Azevedo: \emph{Rolezinho} and cheap illusions]{reinaldo_mistificacoes}}]
		[78
			[{\autocite[][Azevedo: Black blocs and \emph{rolezinhos}]{reinaldo_advogado}}]
			[{\autocite[][Pondé: Double standard]{folha_ponde}}]
		]
	]
	[78
		[100
			[{\autocite[][Azevedo: On \emph{Folha de S. Paulo} survey]{reinaldo_datafolha}}]
			[{\autocite[][Azevedo: Workers' Party and social unrest]{reinaldo_paucome}}]
		]
		[75
			[{\autocite[][Constantino: \emph{Rolezinho} and barbarism]{constantino_barbarie}}]
			[{\autocite[][\emph{Folha de S. Paulo}: 82\% against \emph{rolezinhos}]{folha_datafolha}}]
		]
	]
]
\end{forest}
\caption{The \emph{neocons} subcluster.}
\label{neocons_subcluster}
\end{figure}

Brazilian philosopher Luiz Felipe Pondé, for instance, only refers to \emph{rolezinhos} in a brief moment of his text to expose the hypocrisy of \blockcquote{folha_ponde}[.]{intellectuals who glamorize the \emph{rolezinhos}} The \emph{rolezinhos} are just a side note in an ongoing argument with progressive debaters.

\emph{Rolezinhos} found a more outspoken critic in Rodrigo Constantino, a blogger at Veja magazine. On January \nth{14}, he defined the participants in \emph{rolezinhos} as \blockcquote{constantino_barbarie}{barbarians who are not willing to recognize their own inferiority, who are green with envy of the civilization} (total count: 10,727). Later that same day, in an article titled \enquote{The apartheid in shopping malls is a behavioral one} (total count: 2,471), he wrote \blockcquote{constantino_comportamental}[.]{Yes, there is a class conflict. There is even an apartheid in Brazil today. But it is unrelated to race or skin color. It is a behavioral one! Good, hardworking, taxpaying citizens are victims both of the government that steal their money to pay welfare grants and of crooks who mess with their daily life}

After being heavily criticized on the Internet, he felt like writing a third piece that same day to explain what he meant by \enquote{barbarians who are not willing to recognize their own inferiority.} He explained: \blockcquote{constantino_resposta}[.]{I did not say that every funk lover is an inferior being. \textelp{} I said that someone who organizes a \emph{rolezinho}, as if the mall is a funk party, is giving evidence of envy, of a barbarian, uncivilized, and disrespectful behavior}

In a personal interview for this research, Constantino explained his use of the word \emph{arrastão} to define the \emph{rolezinhos}: \blockcquote{interview_constantino}[.]{The debates on social networks tend to be very polarized. As a right wing blogger in a country where the leftist ideology is hegemonic, I feel the need of reacting in a more emphatic way. If our nation were more civilized, I would prefer to debate in a calmer tone, with fewer adjectives. I understand that I am pushing the envelope when I use the word \emph{arrastão} to define the \emph{rolezinhos}. \emph{Arrastão} presupposes an intention to steal that is usually absent in \emph{rolezinhos}. But in this dichotomous world of the social networks, where the alternative for me is the leftist opinion that \emph{rolezinhos} are inoffensive, I will certainly side with those who call them \emph{arrastão}}

In the \emph{neocons subcluster}, Constantino is very close to the only article that is not a neocon's text. It is a news piece about a \emph{Folha de S. Paulo} survey on how inhabitants of São Paulo---the so-called \emph{paulistanos}---feel about \emph{rolezinhos}, published on January \nth{23}. Naturally, such proximity between Constantino and the survey is not by chance. After reading both texts, one understands how close to each other are \emph{paulistanos}' and Constatino's opinions on \emph{rolezinhos}.

The first paragraph of the survey read like this: \blockcquote{folha_datafolha}[.]{If \emph{rolezinho} is a protest against the social apartheid---as some progressive sectors portray it---, this survey shows that the city of São Paulo is deeply conservative: 82\% of the \emph{paulistanos} are against the gatherings in malls} In the East Side of the city---the poorest region and venue for many \emph{rolezinhos}---, the percentage of disapproval is even higher: 92\%. Around 73\% of the \emph{paulistanos} think the police must act to curb the \emph{rolezinhos}.

Conservative commentators celebrated the results. Reinaldo Azevedo---an influential blogger at Veja magazine and author of four texts in the \emph{neocons subcluster}---wrote a post titled \enquote{The subintellectuals \textins*{are} disappointed with the people}: \blockcquote{reinaldo_datafolha}[.]{While rich leftists engage in a frantic exercise in creative anthropology to see the \emph{rolezinhos} as a scream of the poor against capitalism, those same poor request the end of the youth gatherings in order to go to the malls with no hustle}

Undoubtedly, the term \emph{arrastão} lost currency during the research period, but the spirit of the \emph{arrastão frame} remained stronger than ever.

\section{Social and Racial Apartheid}
\label{sec:apartheid}

The \emph{apartheid cluster} (Figure~\ref{apartheid_frame_annotated}) is the biggest of the three clusters that have been identified as frames.

\begin{figure}
\centering
\resizebox{!}{\dimexpr\textwidth-1.5cm}{
\begin{forest}
[68, for tree={grow=east,anchor=west,child anchor=west}
	[85
		[89
			[{\autocite[][PC Siqueira: \emph{Rolezinhos} are lawful]{youtube_pcsiqueira}}]
			[99
				[{\autocite[][Quinteiro: In defense of \emph{rolezinhos}]{diegoquinteiro}}]
				[{\autocite[][A proto-\emph{rolezinho} in 2000]{documentario_2000}}]
			]
		]
		[{\hyperref[progressivecomentators_subcluster]{\emph{Progressive comentators} subcluster} (Figure~\ref{progressivecomentators_subcluster})}]
	]
	[83
		[{\hyperref[anticonservatives_subcluster]{\emph{Anti-conservatives} subcluster} (Figure~\ref{anticonservatives_subcluster})}]
		[95
			[{\hyperref[anticourtinjunctions_subcluster]{\emph{Anti-court injunctions} subcluster} (Figure~\ref{anticourtinjunctions_subcluster})}]
			[91
				[{\hyperref[antipolice_subcluster]{\emph{Anti-police} subcluster} (Figure~\ref{antipolice_subcluster})}]
				[92
					[89
						[87
							[{\autocite[][Satire with scene of the movie \emph{Downfall}]{hitler_rolezinhos}}]
							[86
								[{\autocite[][Brands ashamed of the poor]{uol_grifes}}]
								[{\autocite[][Brazilian elite is tacky]{folha_elite_classec}}]
							]
						]
						[94
							[{\autocite[][Minister: Violent reaction can worsen the situation]{folha_carvalho}}]
							[87
								[{\autocite[][Minister: White dudes' fears]{folha_medo_brancos}}]
								[93
									[{\autocite[][\emph{Folha}: How the gatherings began]{folha_jovens_fas}}]
									[{\autocite[][Social scientists discuss the \emph{rolezinho}]{elpais_rebeliao}}]
								]
							]
						]
					]
					[97
						[94
							[{\autocite[][\emph{Rolezinho} with college students]{youtube_fea_usp}}]
							[{\autocite[][\emph{Rolezinho} with college students in \emph{Folha de S. Paulo}]{folha_rolezinho_USP}}]
						]
						[86
							[{\autocite[][Pochmann: There is a privatization of public space]{brasildefato_pochman}}]
							[{\autocite[][Protest-\emph{rolezinho} in Cuiabá]{gazeta_digital}}]
						]
					]
				]
			]
		]
	]
]
\end{forest}
}
\caption{The \emph{apartheid} cluster. Collapsed and annoted version of Figure~\ref{apartheid_frame}.}
\label{apartheid_frame_annotated}
\end{figure}

After reading the articles in the cluster, it is possible to describe their common denominator as the belief that \emph{rolezinhos} reveal the cruel nature of the social and racial exclusion of the majority of poor (and usually black) population living in the suburbs of Brazilian larger cities. 

Besides the word \emph{apartheid} (which was in vogue due to Nelson Mandela's death on December \nth{5}), the analogous terms \enquote{prejudice} and \enquote{segregation} were also considered keywords for the computational analysis described in this section.

In fact, many articles do not use the term \emph{apartheid} although they do belong to the \emph{apartheid frame}. Leonardo Sakamoto, an influential progressive blogger, told in a personal interview that he consciously decided to not use the word \emph{apartheid} in his article, although he endorses the view that there is a disturbing element of prejudice in the reaction against \emph{rolezinhos}: \blockcquote{interview_sakamoto}[.]{You cannot vulgarize the words. You have to preserve them for when they are really necessary. In the case of \emph{rolezinhos}, apartheid is too much}

The \emph{apartheid cluster} (Figure~\ref{apartheid_frame_annotated}) is harder to dissect than the other two clusters. There are two hypothesis to explain why. First, the subclusters might not be as cohesive as in the \emph{arrastão} frame. Second, the articles in the \emph{apartheid cluster} might be too homogeneous to allow a clear differentiation among subclusters. In any case, the following criteria was followed: a subcluster is any subtree with at least three leaves and a discernible common argument. Accordingly, four subclusters were identified and will be described in the following. 

\subsection*{Progressives}
\label{sec:progressives}

A quantitative analysis (Figure~\ref{chart_apartheid}) shows how the growth of the \emph{apartheid} frame coincided with the plunge of the \emph{arrastão} frame.

\begin{figure}
\begin{center}
\begin{asy}
real[] percentageOfArrastao = {84.62, 66.67, 47.22, 50.00, 18.18, 22.06, 5.42, 3.14, 2.09, 1.81, 1.24, 0.00, 0.00};
real[] percentageOfApartheid = {0.00, 3.03, 8.33, 22.22, 22.73, 13.24, 29.55, 26.32, 30.11, 15.16, 18.63, 15.00, 10.53};
combo_chart("\% \emph{Arrastão}", percentageOfArrastao,
			"\% Apartheid", percentageOfApartheid);
\end{asy}
\end{center}
\caption{Weekly percentage of stories that mention apartheid, segregation, and prejudice.\label{chart_apartheid}}
\end{figure}

A qualitative analysis of the texts reveals that the concurrence of both trends was not by chance. The texts that support the \emph{apartheid} thesis often establish a polemical dialogue with the \emph{arrastão frame}. A good example is the first text that mentions the term apartheid. That article was published on December \nth{10} by \emph{Baderna Midiática} blog. The title \blockcquote{badernamidiatica}{Brazilian apartheid is exposed. About the so-called \emph{arrastões} in malls} not only questioned the use of the term \emph{arrastão} to define what happened in \emph{Shopping Metro Itaquera} but also it pioneered the use of the word apartheid to explain the phenomenon of \emph{rolezinhos}. \emph{Baderna Midiática} (Media Uproar, in Portuguese) is an activist collective. Half of its roughly 20 members are students in the History department at University of São Paulo.

\emph{Baderna Midiática}'s text carries no byline. However, it was written by historian André Godinho. In a personal interview, Godinho said: \blockcquote{interview_godinho}[.]{The idea of opposing \enquote{apartheid} and \enquote{\emph{arrastão}} in the title came from posts in my Facebook newsfeed by black organizations protesting against the episode of racism in Vitória}

On November \nth{30}, 2013, an incident in Vitória, capital of the state of Espírito Santo, had already revealed how \emph{arrastão} is usually the first analogy used to explain \emph{rolezinho}-like events \autocite{vitoria_arrastao}. A funk party had been organized in the vicinity of \emph{Shopping Vitória}, a local mall. The police decided to investigate what was happening and possibly quit the party. Dozens of youth fled to the mall to avoid problems with the police, but the customers and shopkeepers interpreted the new visitors as an \emph{arrastão} and called the police. There were no thefts or robberies, but many young people were searched. They were forced to take off their t-shirts and sit on the floor with the hands on the head.

A passerby took pictures with his mobile phone and posted on Facebook. The images of black innocent youth subjugated by the police infuriated civil rights organizations. On social networks, they thundered against the use of the term \emph{arrastão} by traditional media and drew a comparison between the incident and the apartheid. Those references inspired Godinho in his text about the first \emph{rolezinho}.

\emph{Baderna Midiática}'s total count was only 798, so it does not appear in the list of 60 top articles that fed the cluster analysis. Nonetheless, one of the subclusters followed in its footsteps: the \emph{progressive commentators subcluster} (Figure~\ref{progressivecomentators_subcluster}). Most of them are well-established left-leaning intellectuals and journalists who work for major media outlets in Brazil.

\begin{figure}
\centering
\begin{forest}
[77, for tree={grow=east,anchor=west,child anchor=west}
	[91
		[{\autocite[][Maria Frô: Bricolage of arguments]{maria_fro}}]
		[90
			[{\autocite[][\emph{G1}: Participants explain the \emph{rolezinhos}]{g1_naspalavras}}]
			[93
				[88
					[{\autocite[][Capriglione: second \emph{rolezinho}]{folha_laura}}]
					[{\autocite[][Martín: Apartheid in the mall?]{elpais_apartheid}}]
				]
				[86
					[{\autocite[][Pinheiro-Machado: opinion from Oxford]{rosana}}]
					[{\autocite[][Minister: white dudes fear \emph{rolezinhos}]{folhapolitica_medo_brancos}}]
				]
			]
		]
	]
	[70
		[93
			[{\autocite[][Boff: We are an unfair society]{boff_rolezinhos}}]
			[87
				[{\autocite[][Brum: New barbarians]{brum_vandalos}}]
				[{\autocite[][Sakamoto: The virtual reality of malls]{sakamoto_rolezinho}}]
			]
		]
		[74
			[{\autocite[][Brum: An homophobic crime and the \emph{rolezinhos}]{brum_kaique}}]
			[{\autocite[][Left-wing party approaches promoters of \emph{rolezinhos}]{terra_rolezeiros_ujs}}]
		]
	]
]
\end{forest}
\caption{\emph{Progressive comentators} subcluster.}
\label{progressivecomentators_subcluster}
\end{figure}

It is easy to see why shopping malls rejected the \emph{arrastão} frame early---they did not want to lose customers---but it is surprising that, in the first week of the controversy, only one article directly challenged the potentially biased nature of that frame. It is likely that many progressive activists and reporters were unsure about how to react to the \emph{arrastão} frame after seeing the contrast between the vehemence of the malls in denying the occurrence of an \emph{arrastão} and the numerous reports on social networks of people who claimed to have witnessed shoplifting. There is, however, a significant increase in the number of mentions of the terms \enquote{apartheid,} \enquote{prejudice,} and \enquote{segregation} in the following weeks of the controversy. They coincide with the second wave of \emph{rolezinhos}.

On December \nth{14}, Saturday, a \emph{rolezinho} in \emph{Shopping Internacional de Guarulhos} \autocite{g1_guarulhos} ended up with 23 people detained \autocite{estado_guarulhos,folha_guarulhos}. Again, the first frame used by the media was the suspicion of \emph{arrastão}. However, in the following day, both the mall and the police asserted that no theft had been registered. The young people who had been brought to the police station were released that same night because the police found no evidence that they had participated in any sort of criminal action \autocite{folha_libertacao_guarulhos}.

On Monday, December \nth{16}, journalist Laura Capriglione, who was present at the mall during the \emph{rolezinho}, published a report in \emph{Folha de S. Paulo} newspaper \autocite{folha_laura}. The article was simply her narrative of the facts but put the young funk fans in a very favorable light. It emphasized from the first line that they have committed no crime---neither theft nor illegal substance possession---and hinted that the police's decision of taking some of them to the police station was arbitrary and disproportionate.

At the same time, it reported with irony the way a small business owner at the mall was \enquote{criticizing \textins*{the youth} among mouthfuls of rump pizza with cream cheese.} References to the flavor of the pizza became a source of jokes and social criticism on Twitter \autocite{twitter_picanha_catupiry} and Capriglione's article received wide dissemination on Facebook (total count: 29,390). The text became the standard narrative for many progressive commentators. Capriglione had on her side the fact that she was the only journalist who witnessed the event. The other media outlets could only count on second-hand narratives from the police and the mall's PRs.

That same day, progressive blogger Leonardo Sakamoto posted a comment \autocite{sakamoto_rolezinho} based on Capriglione's narrative that drew considerable attention (total count: 17,952). It began with the hostile comments by the small business owner who ate rump pizza and then stated that the way the mainstream media was dealing with the \emph{rolezinhos}---the \emph{arrastão} frame---reinforced racial and social stereotypes, and did not provide a reasonable explanation for the gatherings in shopping malls. His reflection ended with vows of more rolezinhos in order to \enquote{see if the bubble \textins*{that isolates the economic and social elite in São Paulo} bursts.}

In an interview for this research project \autocite{interview_sakamoto}, Sakamoto mentioned that, when he wrote the text, he was not so interested in the \emph{rolezinho} issue itself, but how the \emph{rolezinho} exposed another problem: the urban aberration represented by shopping malls in São Paulo. In the absence of enough public parks or theaters, they became the number one alternative for leisure in the city. However, they are not places for culture or sports. The main pleasure they offer is consumption. Sakamoto also guaranteed that he had not read Godinho's article in \emph{Baderna Midiática} blog. Godinho confirmed that he did not have direct contact with anyone in the group of progressive commentators that led the debate after his intervention.

Another article in the corpus \autocite{quadrado_loucos} with less although significant impact (total count: 4,275) and more incisive in its criticism was published that same day in a left-wing website called \emph{Quadrado dos Loucos}. It accused the police and malls of racism, stating that the young people had been taken to the police station \enquote{because they are black.} It also underlined that the use of the word \emph{arrastão} to describe \emph{rolezinhos} was a consequence of the racism pervading Brazilian press and it manifested a perverse habit when it comes to judge any social phenomena involving black poor people. The text was republished on various left-wing websites and blogs.

Those three articles in the second week laid the foundations of the \emph{apartheid frame}. In the following weekend, there were two more rolezinhos: in \emph{Shopping Campo Limpo} \autocite{folha_campo_limpo_1,folha_campo_limpo_2} and \emph{Shopping Interlagos} \autocite{folha_interlagos,g1_interlagos,ig_interlagos,r7_interlagos,estado_interlagos}. On Monday, December \nth{23}, journalist and commentator Eliane Brum published in the Brazilian version of the Spanish newspaper \emph{El País} an article that also drew great attention on the social networks: \blockcquote{brum_vandalos}{The new Brazilian \enquote{thugs}} (total count: 31.405). In that article, after harshly criticizing the discrimination inflicted in young black and poor people, Brum interviews the sociologist Alexandre Pereira Barbosa, a specialist in \enquote{cultural manifestations from São Paulo's suburbs.}

One week later, on December \nth{30}, Rosana Pinheiro-Machado, Professor of Anthropology of Development at University of Oxford, contributed to the debate with an article titled \blockcquote{rosana}{Ethnography of Rolezinho} published on her own blog (total count: 26,574). From then on, she would become one of the most quoted experts on the subject of \emph{rolezinhos}. She argued that \emph{rolezinhos} are not traditional protests, but they clearly reveal the exclusion and discrimination against young people in the suburbs.

Those handful of articles helped to set the tone of the debate before the media storm in the third week of January. Later on, other commentators worked to keep the flame burning. On January \nth{23}, for instance, Leonardo Boff published \blockcquote{boff_rolezinhos}[.]{The rolezinhos blame us: we are an unjust and segregationist society} He mentioned \citeauthor{rosana} as one of his references. With a remarkable total count of 35,688, Boff's text is the ninth most popular article. He is an important thinker of the Liberation Theology, an influential movement among Latin American Christians, chiefly Catholics, that advocates for social justice and has a Marxist inspiration.

On January \nth{13}, activist Maria Frô published an article \autocite{maria_fro} (total count: 8,533) that criticized \enquote{malls, the bourgeoisie, the police, and the Courts.} It referred to the police as \emph{capitão do mato} (literally, \enquote{captain of the woods})---the \nth{19} Century term for those who performed the vile task of recapturing escaped slaves---which is a clear reference to the racist component in the reaction to \emph{rolezinhos}.

The \emph{progressive commentators subcluster} received 10.6\% of the total Facebook attention in the corpus. As a comparison, it is more than three times the amount their nemeses---the \emph{neocons}---gathered (2.94\%).

\subsection*{Anti-police and anti-injunctions}

The \emph{apartheid} frame received new impetus after several shopping malls obtained court injunctions to prevent \emph{rolezinhos} \autocite{folha_liminar_shoppings,estado_liminar_shoppings,veja_liminar_shoppings,g1_liminar_shoppings,band_liminar_shoppings} and after the violent police response on January \nth{11} at the Shopping Metro Itaquera \autocite{band_confronto_itaquera,folha_confronto_itaquera,ig_confronto_itaquera,estado_confronto_itaquera,g1_confronto_itaquera,r7_confronto_itaquera}. According to the reports, police expelled about a thousand people from the shopping mall and then used tear gas and rubber bullets to disperse them. After that incident, the share of news related to the \emph{apartheid frame} more than doubled (from 13\% to 30\%) according to our corpus (cf. Figure~\ref{chart_apartheid}).

On January 13th, Brazilian writer Vanessa Barbara, who witnessed the police's action at the mall, published her report in \emph{Folha de S. Paulo} \autocite{folha_vanessa}. According to her, the police tried to intimidate the youth with threats such as \enquote{I'll break your face.} That week, Barbara's article had a significant impact (total count: 10,632) and played a similar role to Laura Capriglione's story \autocite{folha_laura} in December, offering to progressive commentators a convenient narrative to criticize the crackdown on \emph{rolezinhos}. Two days early, \emph{Folha de S. Paulo} had already published a video that showed policemen assaulting youth in \emph{Shopping Metrô Itaquera} \autocite{folha_pms_agredindo} (total count: 13,435).

Both articles belong to another group of URLs: the \emph{anti-police subcluster} (Figure~\ref{antipolice_subcluster}). It is a small set of articles that conveys a very critical position about the police.

\begin{figure}
\centering
\begin{forest}
[86, for tree={grow=east,anchor=west,child anchor=west}
	[{\autocite[][Neder: Against the State violence]{neder}}]
	[95
		[{\autocite[][\emph{Folha de S. Paulo}: Police beats protesters]{folha_pms_agredindo}}]
		[{\autocite[][Barbara: Witnessing violence]{folha_vanessa}}]
	]
]
\end{forest}
\caption{\emph{Anti-police} subcluster.}
\label{antipolice_subcluster}
\end{figure}

In addition to the flood of articles criticizing the Brazilian apartheid, \enquote{protest-\emph{rolezinhos}} were organized by social movements \autocite{estado_rolezinho_leblon, folha_rolezinho_leblon, veja_rolezinho_passeata}, especially black organizations \autocite{folha_protesto_jk} and the Homeless Workers' Movement \autocite{folha_semteto}. Unlike traditional \emph{rolezinhos}, those protests happened inside (or at least in front of) shopping malls in affluent neighborhoods. While the organizers of the traditional \emph{rolezinhos} tried to minimize conflict with security guards and the police, the protesters tended to stimulate confrontation in order to drew more attention to their agenda. Their political message was also much clearer, because it was stated in their flyers and banners. Such protesters helped to spread the \emph{apartheid} frame in the media coverage because it was unavoidable to mention their demands when reporting the protests.

Such demonstrations were fueled by the revolt against the court injunctions that threatened with fine and detention those who organized and attended to \emph{rolezinhos}. One article that became the fourth most popular written piece in the controversy had the following headline: \blockcquote{veto_consagra_apartheid}{Prohibition of rolezinho enshrines the Brazilian apartheid} (total count: 53,480). Published on the left-wing portal \emph{Brasil 24/7} on January \nth{12}, it drew a parallel with the segregation that African-Americans suffered in the United States in the 60s. It even embedded a YouTube video with a civil-rights discourse by president John F. Kennedy. That article belongs to another small subcluster: the \emph{anti-injunctions} (Figure~\ref{anticourtinjunctions_subcluster}), a group of texts that criticized the courts for giving institutional support to an allegedly racist action.

\begin{figure}
\centering
\begin{forest}
[83, for tree={grow=east,anchor=west,child anchor=west}
	[{\autocite[][Prohibition of \emph{rolezinho} enshrines apartheid]{veto_consagra_apartheid}}]
	[99
		[{\autocite[][The right to organize a \emph{rolezinho}]{blogdacidadania}}]
		[{\autocite[][Prohibition and right to segregation]{folha_direito_segregacao}}]
	]
]
\end{forest}
\caption{\emph{Anti-court injunctions} subcluster.}
\label{anticourtinjunctions_subcluster}
\end{figure} 

\subsection*{Anti-conservatives}

The main criticism against the \emph{apartheid} frame was based on the fact that the traditional \emph{rolezinhos} did not occur in affluent malls. They were born and spread in the malls located in the poorest regions of São Paulo, enterprises that thrived thanks to the new prosperity and generous credit supply to the low income families during the last decade.

\emph{Shopping Metro Itaquera}, for instance, is located in a neighborhood where 58\% of the population belongs to low-income families and 34\% earn less than \$320 per month (the average percentages for São Paulo are 53.1\% and 27.8\%, respectively) \autocite{dnapaulistano}. \emph{Shopping Internacional de Guarulhos} is based on the second most populous city of the state of São Paulo, with about 1.3 million people. According to a 2003 study by the Brazilian Institute of Geography and Statistics (IBGE) about 43.21\% of the population in Guarulhos is below the poverty line, a percentage well above the State average (26.6\%) and even the city of São Paulo (28\%) \autocite{mapapobreza}.

The first article that highlighted that possible contradiction was published on December \nth{19} in a controversy against Leonardo Sakamoto's article \autocite{sakamoto_rolezinho} of December \nth{13}. The title was \blockcquote{luciano_ayan}[.]{Sakamoto and rolezinhos or \enquote{another evidence that the Left hates the honest poor}} That post in Luciano Ayan's blog \emph{Ceticismo Político} (Political Skepticism) did not receive much attention (total count: 261). The argument was: we are not watching a class conflict, but a struggle between poor people who want a funk party at the mall and poor people who want peace and safety to go shopping and eat BigMac's.

Conservative commentator Reinaldo Azevedo also published two articles---\blockcquote{reinaldo_mistificacoes}{Rolezinho and cheap illusions} in \emph{Folha de S. Paulo} (total count: 5,118) and \blockcquote{reinaldo_esquerda}{The foolish Left already wants to take advantage of the rolezinho} in Veja magazine's Website (total count: 4,156)---along the same lines but with more impact.

The \emph{apartheid frame} answered to those criticisms through theoretical arguments---represented by the \emph{progressive commentators subcluster}---and through ad hominem attacks that constituted a subcluster on its own (Figure~\ref{anticonservatives_subcluster}). The strategies ranged from ridiculing the adversary with satire or irony to proposing legal actions to silence them.

\begin{figure}
\centering
\begin{forest}
[98, for tree={grow=east,anchor=west,child anchor=west}
	[{\autocite[][Against Sheherazade]{cartacapital_sheherazade}}]
	[70
		[{\autocite[][Satire against Constantino and Azevedo]{piaui_rolezinho_miami}}]
		[71
			[{\autocite[][Amorim: Irony against \emph{Folha de S. Paulo}]{pha}}]
			[{\autocite[][\emph{Rolezinho} in Brasilia]{folhapolitica_rolezinho_congresso}}]
		]
	]
]
\end{forest}
\caption{\emph{Anti-conservatives} subcluster.}
\label{anticonservatives_subcluster}
\end{figure}

Lino Bocchini, for instance, suggested that the government should revoke the broadcasting concession for SBT, the TV station that aired Rachel Sheherazade's commentaries \autocite{cartacapital_sheherazade}. Another text in the corpus has the title \blockcquote{elite_podre}[.]{\textins*{Rodrigo} Constatino: symbol of a rotten elite}

The former Globo reporter and now enthusiast of the Workers' Party Paulo Henrique Amorim published a very successful 33-second video \autocite{pha} (total count: 26,021) on January \nth{15}. He reads an excerpt of a \emph{Folha de S. Paulo} analysis that foresaw \blockcquote{folha_analise_gripp}[.]{the rolezinhos would lose their original meaning \textins{of a poor youth party} and become more similar to the June protests of the last year} Amorim finishes with an ironic smile. The title of the video gives the key to understand his irony: \enquote{\emph{Folha de S. Paulo} summons and organizes \emph{rolezinhos}.} The implicit part is: \enquote{\ldots{}in an attempt to destabilize president Dilma's government.}

\section{Middle ground}

There is a third and small frame that challenges both the \emph{apartheid} and the \emph{arrastão} perspectives. It does not criminalize the youth but, at the same time, sympathizes with those who do not agree with \emph{rolezinhos} in the mall. The chosen label for it is \emph{middle ground} (Figure~\ref{middleground_frame_annotated}).

\begin{figure}
\centering
\begin{forest}
[98, for tree={grow=east,anchor=west,child anchor=west}
	[{\autocite*[][Cauê Moura on \emph{rolezinhos}]{youtube_caue_moura}}]
	[98
		[{\autocite[][Beguoci: \emph{Rolezinhos} and dehumanization]{beguoci_rolezinhos}}]
		[{\autocite[][What about a \emph{rolezinho} in the library?]{revista_bula}}]
	]
]
\end{forest}
\caption{The \emph{middle ground} cluster. Annoted version of Figure~\ref{middleground_frame}.}
\label{middleground_frame_annotated}
\end{figure}

The \emph{middle ground cluster} comprises only three texts (out of 60) in the top article sample. At first sight, it might seem an insignificant share: only 5\% of the top articles corpus. However, it represents 13\% of the Facebook attention in that group and 7\% in the overall corpus.

In fact, the most popular article in the corpus belong to the \emph{middle ground cluster}: a seven-minute Youtube video by Cauê Moura in a channel called \emph{Desce a letra} \autocite{youtube_caue_moura} (total count: 86,228). Published on January \nth{13}, therefore at the peak of the media storm, it is an unashamedly didactic piece that relies on exotic look and aggressive language to attract teen attention (which probably explains his video's huge popularity). According to Moura, his Youtube channel aims to help adolescents think more critically about contemporary issues \autocite{portal_imprensa_caue_moura}. 

In his commentary on the rolezinhos, Moura points out that most \emph{rolezinhos} happen in the suburbs. That fact, according to him, renders the \enquote{class conflict discourse} void. At the same time, he says it is impossible to deny the existence of prejudice against poor black people. He tries to delineate an alternative frame to the \emph{arrastão} and \emph{apartheid} narratives, but notwithstanding castigating the right and left-wing readings of the controversy, the video does not try to offer a full-fledged explanation of the possible causes of the \emph{rolezinhos}.

A more nuanced analysis came with Leandro Beguoci's article \blockcquote{beguoci_rolezinhos}{Rolezinho and the dehumanization of the poor} published on the next day, January \nth{14}. The subtitle gives a good summary: \enquote{The extremist debates by the Left and the Right are ignoring the people who take part in the rolezinho. It is time to understand the suburbs.} Beguoci, a former reporter at \emph{Folha de S. Paulo}, drew on his personal experiences as a poor teenager in the Greater São Paulo and on his reflections of the recent social changes in Brazil to reach the conclusion that the hopes and fears related to the \emph{rolezinhos} are, by and large, unrealistic.

In a personal interview for this research, Beguoci explained the reasons that moved him to write that article: \blockcquote{interview_beguoci}[.]{When I read Laura Capriglione's article \autocite{folha_laura} on the second rolezinho, something bothered me\ldots{} It was probably the depiction of the small business owner criticizing the youth \enquote{among mouthfuls of rump pizza}. That narrative leads to think that she belongs to a distant elite, but I don't think that is the case. She could be one of the participant's mother} At the same time, he was far from supporting the hostility that some conservative commentators expressed towards the participants in \emph{rolezinhos}. According to him, his article was an attempt at offering a third perspective outside the realm of the cultural wars.

He invites the reader to follow the example of Samuel Klein, a Brazilian entrepreneur who realized that poor people are often better customers than the wealthier consumers and founded a successful retail chain in poor neighborhoods. According to Beguoci, Klein was a visionary leader because he left behind sociological discourses and elitist prejudices, and decided to talk to poor people.

Beguoci affirms that he did not draw inspiration from specific articles, although he appreciated Vinicius Torres Freire's \blockcquote{folha_vinicius}{Um rolê pelo rolezinho} in \emph{Folha de S. Paulo} (total count: 307). According to Beguoci, Freire's text was published on January \nth{14}, when Beguoci's article was almost finished. Freire described the \emph{rolezinho} issue as a \enquote{stereotypical conflict between the left and the right} over the real meaning of the youth gatherings.

Beguoci's article was published in a young media venture called \emph{Oene} that has the challenging aim of making money from high quality journalistic content and comments. Significantly, \emph{Oene}'s motto is \enquote{Very reasonable} and the \enquote{About us} section \autocite{oene_sobre} of the Website states that the texts will be as \enquote{objective} as possible and removed \enquote{from the passions that can inspire good chats in a pub but so often blind us.} Beguoci is one of the initiative's founders.

His article reached a remarkable popularity, not only for the number of likes or shares (total count: 45,980) but especially for the prestige of the media outlets that quoted him, as a search for his name in the Media Cloud corpus shows. On the following Sunday, January \nth{19}, \emph{Folha de S. Paulo}'s ombudsman Suzana Singer praised his text as a \blockcquote{folha_ombudsman}[.]{very lucid analysis} That same day, an article in the \emph{The New York Times} \autocite{nyt_rolezinhos}---that would be translated and published on the next day by \emph{O Estado de S. Paulo} newspaper \autocite{estadao_nyt}---also quoted his text. He was also interviewed by TV Cultura \autocite{youtube_beguoci}---the public TV channel in São Paulo---on January \nth{17} and took part in a conference organized by the Cultural Center Ruth Cardoso on February \nth{13} \autocite{youtube_ruthcardoso}. His article's success---2.3\% of the overall attention in the corpus---probably indicates that there is a demand for conciliatory discourses in the Brazilian media ecosystem.

The \emph{middle ground} frame offers a challenge for computational analysis because it is very difficult to associate keywords to it. It blends terms from the \emph{arrastão} frame and the \emph{apartheid} frame.

Apart from that, in the highly polarized debate on the \emph{rolezinhos}, there were not many \emph{middle ground} articles to be found. In the list of top articles, for instance, there is only more one exemplar: Everth Vêncio's \blockcquote{revista_bula}{Rolezinho in the library} (total count: 7,716). In the rest of the corpus, there is another one: Ruth de Aquino's \blockcquote{epoca_aquino}{Summer's rolezão} (total count: 279), published in \emph{Época} magazine on January \nth{17}. She downplayed the importance of the youth gatherings and invited \enquote{leftists and fascists} to quit the hysteria.

\section{Other actors and frames}

Two social actors conveyed coherent messages during the controversy. However, their points of view were usually subsumed in the frames described above and never reached the state of a full-fledged frame. Those are the shopping malls and the politicians---especially the ones in the Workers' Party.

In the analysis so far, the mall's perspective was put under the umbrella of the \emph{arrastão frame}. Although they denied the occurrence of \emph{arrastão}, they supported other dimensions of this frame: the role of the police and the inconvenience of \emph{rolezinhos} for the sake of safety. At the same time, the Workers' Party's take on the \emph{rolezinho} became blended with the \emph{apartheid frame}.

In this section, we are going to analyze the particular elements and strategies employed by those two groups of social actors.

\subsection{Shopping malls}

The shopping malls presented consistent behavior during the span of the controversy. All the time, they were guided by three principles:
\begin{itemize}
\item Shopping malls are private places, so they have the right to expel those who do not behave well.
\item \emph{Rolezinhos} have a perverse economic impact in taxpaying businesses.
\item \emph{Rolezinhos} threaten other customers because shopping malls do not have the infrastructure for that sort of social gathering.
\end{itemize}

Two associations represented mall owners' and storekeepers' rights and demands: the Brazilian Association of Mall's Storekeepers (Alshop) and the Brazilian Association of Shopping Malls (Abrasce). Both organizations were in perfect consonance with each other. The shopping malls also showed a high compliance with the associations' strategies during the media storm.

Nevertheless, only one text among the top articles sample can be considered a pure manifestation of the shopping mall's position: \emph{Folha de S. Paulo}'s piece on the meeting of the president of Alshop, Nabil Sahyon, and São Paulo governor Geraldo Alckmin \autocite{folha_shoppings_rolezodromos}. The text solely conveys Sahyon's point of view that \blockcquote{folha_shoppings_rolezodromos}[.]{all are welcome, but the admission of thousands of people jeopardizes the security of the malls. There may be an accident and the mall will be held accountable. Shopping is not the proper place for a funk party}

In the first week of the controversy, they sent a letter to the State Department of Public Security requesting special policing around the shopping malls \autocite{uol_policiamento_shopping}. To the press, although denying the occurrence of an \emph{arrastão}, they argued that shopping malls are like soccer stadiums and therefore require special protection from the public security services in order to shun any threat to public safety \autocite{estado_novas_invasoes}. To look at the aggressive tactics used by the police on December \nth{14}, with the detention of 23 youth \autocite{folha_guarulhos}, it seems their request was heard by the State government.

However, after two \emph{rolezinhos} over the weekend before Christmas \autocite{folha_campo_limpo_1, folha_campo_limpo_2, folha_interlagos,g1_interlagos,ig_interlagos}, one article in \emph{Folha de S. Paulo} reports that Alshop was trying to lobby for policing inside the shopping malls as a preventive measure \autocite{folha_pm_shopping}. The malls were willing to pay an additional stipend for the policemen. In the same article, São Paulo governor, Geraldo Alckmin, argued against the solution pointing out that shopping malls are private places and must be protected by private guards, not public police.

After a period of calmness during the holidays, on January \nth{11}, various media sources reported the new strategy of the shopping malls: court injunctions (in Portuguese, \emph{liminares}) that would forbid any \emph{rolezinho}. Offenders would be liable for a fine of 10,000 Brazilian reais (\$4,500). The Media Cloud corpus shows that all stories published on January \nth{11}, before the violent repression at \emph{Shopping Metrô Itaquera}, were reproductions of only two articles distributed by the news agencies associated to \emph{Folha de S. Paulo} and the web portal \emph{Terra}. Both pieces have different takes on the issue. \emph{Folha de S. Paulo}'s article \autocite{folha_liminar_shoppings} simply reports that four shopping malls obtained court injunctions to avoid \emph{rolezinhos} scheduled for that weekend. \emph{Terra} preferred to focus on just one of the malls---the Shopping JK Iguatemi---the only one located in a wealthy neighborhood. Its headline highlighted that fact: \blockcquote{terra_liminares}[.]{Luxurious mall obtain court injunction against rolezinho}

As it can be seen in Figure~\ref{chart_courtinjunctions}, the media storm comes just after the soaring of mentions to \enquote{court injunctions.}

\begin{figure}
\begin{center}
\begin{asy}
real[] percentageCourtInjunctions = {0.00, 0.00, 0.00, 0.00, 0.00, 47.06, 30.69, 15.84, 11.92, 9.75, 3.11, 1.43, 2.11};
combo_chart("\% Court injunctions", percentageCourtInjunctions);
\end{asy}
\end{center}
\caption{Weekly percentage of stories that mention \enquote{court injunctions}.\label{chart_courtinjunctions}}
\end{figure}

Such proximity is not a coincidence: 81 out of 199 articles (41\%) that explicitly reflect on the social and racial apartheid have references to the court injunctions. Shopping malls and judges faced an angry backlash from human rights organizations, left-wing parties and progressive commentators over the fairness of the court decisions (see \autoref{sec:apartheid} \nameref{sec:apartheid} on page \pageref{sec:apartheid}). 

The shopping malls answered that they were not discriminating people according to their race or social position, but it was a decision based on safety concerns.

However, the \enquote{everybody is welcome as long as embrace the rules} argument also received an embarrassing response: a YouTube video (total count: 25,619) that shows a messy party in \emph{Shopping Eldorado} that has been organized every year since 2007 by middle to upper class students from the School of Economics at the University of São Paulo \autocite{youtube_fea_usp}. It resembles a \emph{rolezinho} in its crowded and noisy exuberance but without any complaints about security threats or peace disturbance. On the contrary, some shops in the mall even give financial support to it.

It is a particularly interesting case because the video was discovered by the left-wing publication \emph{Rede Brasil Atual} and found its way to the mainstream media. The video was first mentioned on January 13th, in an article polemically titled \blockcquote{rba_fea}{Shopping mall bans youth from the suburbs, but welcomes \emph{rolezinho} with University of São Paulo students} (total count: 3,334). The article was reproduced by other left-wing websites \autocite{altamiro_fea,brasil247_fea} and, on January \nth{15}, the students from the School of Economics at the University of São Paulo answered to it on their Facebook page \autocite{facebook_fea}. Funnily enough, they agreed with the criticism and suggested that there should be equal treatment to their party and to the \emph{rolezinhos}. On January \nth{18}, the mainstream portal \emph{UOL} published a similar article \autocite{uol_fea} (total count: 1,201) about the same topic and, on January \nth{21}, \emph{Folha de S. Paulo} also spread the word \autocite{folha_rolezinho_USP} about the \enquote{upper class rolezinho} (total count: 9,852).

Towards the end of the controversy, the shopping malls decided to adopt a more conciliatory approach \autocite{folha_reuniao_shoppings} suggesting that \emph{rolezinhos} should happen in open spaces like public parks \autocite{folha_alshop_organizacao}.

\subsection{The politics of \emph{rolezinhos}}

Many commentators pointed out that the current \emph{rolezinhos} conflict between the \enquote{traditional middle class} and the \enquote{new middle class} that invaded the shopping malls would never happen without those socioeconomic changes over president Lula era (see \autoref{historical_background} \nameref{historical_background} on page \pageref{historical_background}). According to that interpretation (favorable to Lula's ruling party), the \emph{rolezinhos} would be a \enquote{blessed} problem: a sign that Brazil has been winning the war on poverty.

At least two articles at the beginning of the controversy briefly mentioned the \enquote{new middle class} frame: the already cited \citeauthor{sakamoto_rolezinho}'s \blockcquote{sakamoto_rolezinho}{Rolezinho} and \citeauthor{brum_vandalos}'s \blockcquote{brum_vandalos}[.]{The new Brazilian \enquote{thugs}} After them, other commentators also inserted the \emph{rolezinho} issue in the context of the major economic changes of the last decade \autocite{beguoci_rolezinhos, brasildefato_pochman, maria_fro}.

However, it was only after the incident with rubber bullets and tear gas on January \nth{11} that the \emph{rolezinhos} became a disputed issue between the two main parties in Brazil: the Brazilian Social Democratic Party (PSDB) and the Workers' Party (PT). Figure~\ref{chart_politics} is a comparison of the number of mentions to authorities in the city, state and federal levels.

\begin{figure}[t]
\begin{center}
\begin{asy}
real[] percentageOfMayor = {0.00, 3.03, 38.89, 22.22, 40.91, 14.71, 6.85, 8.50, 9.18, 3.61, 13.04, 1.43, 1.05};
real[] percentageOfGovernor = {0.00, 0.00, 0.00, 5.56, 0.00, 2.94, 11.28, 8.83, 4.19, 5.05, 3.11, 2.14, 5.26};
real[] percentageOfPresident = {0.00, 0.00, 2.78, 0.00, 0.00, 0.00, 17.06, 23.02, 23.67, 12.27, 15.53, 11.43, 10.53};
combo_chart("\% Mayor", percentageOfMayor,
			"\% Governor", percentageOfGovernor,
			"\% President", percentageOfPresident);
\end{asy}
\end{center}
\caption{Weekly percentage of stories that mention officials in the city, state, and federal levels.\label{chart_politics}}
\end{figure}

In the city level, two authorities were considered: the mayor Fernando Haddad (PT) and Netinho de Paula, who was chosen by the mayor to negotiate with the organizers of the \emph{rolezinhos}. It is clear that they have a more prominent role in December, but not during the media coverage peak, according to Figure~\ref{chart_politics}.

For the state level, the governor Geraldo Alckmin (PSDB) and the Secretary of Public Security, Fernando Grella, were included in the analysis. Their importance in the debate increased significantly after the incident on January \nth{11}, most of the time because they were blamed (and sometimes praised) for the police reaction to the \emph{rolezinho}.

However, the increase in mentions of officials in the federal level is impressive after January \nth{11}. In addition to president Dilma Rousseff (PT), the query considered the minister of Racial Equality Luiza Bairros and the minister of the General Secretariat of the Presidency, Gilberto Carvalho. Until then, there had been almost no mentions to any federal authority.

On January \nth{16}, Bairros' opinion on the \emph{rolezinhos} made the headlines: \blockcquote{folha_medo_brancos}{Fear of \emph{rolezinhos} is reaction of white dudes} (total count: 13,112). That same day Carvalho attacked the aggressive tactics used by São Paulo state police and said that it was like \blockcquote{folha_carvalho_fogo}[,]{throwing gasoline in a bonfire} meaning that it would make the social unrest even worse (total count: 5,269).

Interestingly, the main reaction did not came from the governor or his party, but from the mainstream media. In the following day, \emph{O Estado de S. Paulo} newspaper reported Bairros' and Carvalho's opinions under the headline: \blockcquote{estado_rolezinho_oposicao}[.]{\textins{President} Dilma is already using the \enquote{rolezinhos} against the opposition} (total count: 1,347). The article affirmed that both ministers were following Dilma's guideline of using the \emph{rolezinho} issue to attack governor Geraldo Alckmin (PSDB) and weaken his party for the second semester elections.

Conservative journalist Reinaldo Azevedo also posted a harsh criticism on his blog at \emph{Veja} magazine: \blockcquote{reinaldo_carvalho}{An absurd without Gilberto Carvalho is always incomplete} (total count: 2,637). He accused Carvalho of promoting \emph{rolezinhos} with his criticisms against the São Paulo police.

\emph{O Globo}---an influential newspaper that is part of the most important media conglomerate in Latin America---also published an editorial on January \nth{21}. The title was \blockcquote{editorial_globo}[.]{\emph{Rolezinhos} and manipulations} Among other things, it accused the federal government of \enquote{creating problems} to governor Geraldo Alckmin (PSDB), \enquote{the enemy to be defeated in the ballot box this year.}

Eventually, when the media spotlight moved away from the \emph{rolezinhos}, the topic also disappeared from the officials' discourses.