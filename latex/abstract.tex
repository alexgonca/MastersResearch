This research analyzes the battle of frames in the controversy surrounding \emph{rolezinhos}---flashmobs organized by low-income youth in Brazilian shopping malls. To analyze framing of these events, a corpus of $5,000$ online articles was analyzed. These articles, published between December \nth{7}, 2013, and February \nth{23}, 2014, were investigated using Media Cloud---the system for large scale content analysis developed by the Berkman Center at Harvard and the MIT Center for Civic Media. Data from Facebook indicated which articles received more attention on the social network.

A framing analysis was performed to describe the conflicting frames in the debate. The 60 most popular texts---those that attracted $60\%$ of the social media attention in the corpus---were content analyzed. They served as an input for a hierarchical cluster analysis algorithm that grouped articles with similar frame elements.

The result of the cluster analysis led to the identification of three frames: one that criminalized \emph{rolezinhos} or at least tried to discourage them (\emph{arrastão} frame), another that acquitted the youth and blamed police, government, State, or society for discriminating poor citizens (\emph{apartheid} frame), and a third frame that criticized both conservatives and progressives for using the controversy to push their particular agendas (\emph{middle ground} frame).

After finding the keywords that singled out each frame, natural language processing methods helped to describe the genesis and evolution of the frames in the overall corpus as well as the framing strategies of the main actors.