\chapter{Theories and Methods}

\section{Theoretical background}

In everyday life, we experience a plethora of different stimuli. In order to make sense of them, we rely on \blockcquote[21]{goffman1986frame}[.]{one or more frameworks or schemata of interpretation \textelp{} that \textins{are} seen as rendering what would otherwise be a meaningless aspect of the scene into something that is meaningful} Each schema of interpretation \blockcquote[21]{goffman1986frame}[.]{allows its user to locate, perceive, identify, and label a seemingly infinite number of concrete occurrences \textelp{}} For \citeauthor{goffman1986frame}, the examination of those interpretive frameworks that organize our ordinary experiences constitutes the research method known as \emph{frame analysis} \autocite[11]{goffman1986frame}.

\citeauthor{gitlin1980whole} draws on \citeauthor{goffman1986frame}'s reflections to affirm that \blockcquote[6--7]{gitlin1980whole}[.]{we frame reality in order to negotiate it, manage it, comprehend it, and choose appropriate repertories of cognition and action} Applying the \emph{frame} category to media studies, \citeauthor{gitlin1980whole} defines \emph{media frames} as \blockcquote[7]{gitlin1980whole}[.]{persistent patterns of cognition, interpretation, and presentation, of selection, emphasis, and exclusion, by which symbol-handlers routinely organize discourse} They \blockcquote[7]{gitlin1980whole}[.]{enable journalists to process large amounts of information quickly and routinely: to recognize it as information, to assign it to cognitive categories, and to package it for efficient relay to their audiences}

According to \citeauthor{gitlin1980whole}, \emph{media frames} are unavoidable even if for \enquote{organizational reasons alone}. Nevertheless, they are far from harmless. In a seminal study, \citeauthor{tversky1981framing} present evidence that the same problem framed in different ways can elicit totally distinct decisions \autocite{tversky1981framing}. Such results lend support to \citeauthor{edelman1993contestable}'s depiction of the \enquote{social world} as \blockcquote{edelman1993contestable}[.]{a kaleidoscope of potential realities, which can be readily evoked by altering the ways in which observations are framed and categorized}

Besides the cognitive and organizational justifications for frame selection, \citeauthor{entman1993framing} proposes a deliberate component in framing strategies. For him, \blockcquote{entman1993framing}[.]{to frame is to \emph{select some aspects of a perceived reality and make them more salient in a communicating text, in such a way as to promote a particular problem definition, causal interpretation, moral evaluation, and/or treatment recommendation} for the item described}

For Entman, the promotion of a specific frame is not an unintended side effect, but the core aim of choosing an \blockcquote{garrison1994changing}[.]{organizing idea or story line that provides meaning to an unfolding strip of events}

\citeauthor{entman2007framing} goes even further and points out that \blockcquote{entman2007framing}[?]{\textins*{p}owerful players devote massive resources to advancing their interests precisely by imposing \textins{persistent and politically relevant} patterns on mediated communications. To the extent we reveal and explain them, we illuminate the classic questions of politics: who gets what, when, and how~\autocite{lasswell1950politics}}

For the purposes of this research, the definition of frame is based on \citeauthor{entman1993framing}'s above-mentioned fourfold function: full-fledged frames typically perform problem definition, causal analysis, moral judgment, and remedy promotion~\autocite{entman2007framing}. In addition to being a widely discussed definition of frame in media studies~\autocite{matthes2009what}, it is particularly helpful to consider the agency of journalists and the role of frame sponsorship, oft-neglected dimensions in framing research~\autocite{carragee2004neglect}.

The impact of news framing is not restricted to inducing specific interpretations and reactions. Based on experimental data, \citeauthor{price1997switching} suggest that a frame may stimulate \blockcquote{price1997switching}[.]{a kind of hydraulic pattern, with thoughts of one kind \textelp{} driving out other possible responses} In other words, frames can successfully inhibit competing frames.

\citeauthor{gitlin1980whole} has explored the connection between \emph{media frames} and the Gramscian concept of hegemony---\blockcquote[][I \S 48]{gramsci1975quaderni}{rule by permanently organized consent}---to unravel the relationship between the framing process and the distribution of social and political power. In his analysis of the parliamentary regime, \citeauthor{gramsci1975quaderni} states that \blockcquote[][I \S 48]{gramsci1975quaderni}[.]{\textins*{t}he \enquote{normal} exercise of hegemony \textelp{} is characterized by a combination of force and consent which balance each other so that force does not overwhelm consent but rather appears to be backed by the consent of the majority, expressed by the so-called organs of public opinion} \citeauthor{gitlin1980whole} explicitly refers to Gramsci's core conception and advances it: \blockquote[{\autocites(emphasis in the original)[10]{gitlin1980whole}}][.]{\textins*{T}hose who rule the dominant institutions secure their power in large measure directly \emph{and indirectly}, by impressing their definitions of the situation upon those they rule and, if not usurping the whole of ideological space, still significantly limiting what is thought throughout the society}

To borrow \citeauthor{hallin1989uncensored}'s model of three spheres, thoughts sanctioned by the hegemonic establishment constitute the spheres of consensus and legitimate controversy. Meanwhile, dangerous thoughts that challenge the sociopolitical order are banned to the sphere of deviance and do not gain currency in the public debate. \blockcquote[117]{hallin1989uncensored}[.]{\textins{Journalism} plays the role of exposing, condemning, or excluding from the public agenda those who violate or challenge the political consensus. It marks out and defends the limits of acceptable political conflict}

There have been many attempts of integrating framing studies into the more established scholarship on agenda-setting and priming. Classical agenda-setting theory, as set forward by \citeauthor{mccombs1972agenda}, suggests that the press \blockquote[{\autocites[13]{cohen1965press}[as cited in][]{mccombs1972agenda}}][.]{may not be successful much of the time in telling people what to think, but it is stunningly successful in telling its readers what to think \emph{about}} Priming is, in turn, a specific dimension of agenda-setting: by attending to some problems and ignoring others, media alters the standards by which people evaluate public officials, government, and other institutions \autocite{iyengar1982experimental}.

\citeauthor{mccombs1997candidate} propose that framing can be considered a second level of agenda setting. The first and traditional level states that objects (or issues) emphasized on the media agenda come to be regarded as important on the public agenda. The second level---that the authors identify with framing---\blockcquote{mccombs1997candidate}[.]{is the selection of a small number of attributes for inclusion on the media agenda when a particular object is discussed} They revise the aforementioned \citeauthor{cohen1965press}'s dictum: \blockcquote{mccombs1997candidate}[.]{Explicit attention to the second level of agenda setting further suggests that the media also tell us \emph{how to think} about some objects}

However, some researchers challenge the idea that framing is just an extension of the agenda-setting model \autocite{price1997news,scheufele2007framing}. For instance, \citeauthor{price1997news} argue that \blockcquote[184]{price1997news}[.]{\textins*{a}genda setting looks on story selection as a determinant of public perceptions of issue importance and, indirectly through priming, evaluations of political leaders. Framing focuses not on which topics or issues are selected for coverage by the news media, but instead on the particular ways those issues are presented}

According to \citeauthor{entman2007framing}, notwithstanding the possible divergences on their theoretical premises, both agenda-setting and framing strategies are closely united in the communication arsenal of political, cultural, and economic elites: \blockcquote{entman2007framing}[.]{Given limitations of time, attention, and rationality, getting people to think (and behave) in a certain way requires selecting some things to tell them about and efficiently cueing them on how these elements mesh with their own schema systems. Because the best succinct definition of power is the ability to get others to do what one wants \autocite{nagel1975power}, \enquote{telling people what to think about} \textins{the classical definition of the agenda-setting effect} is how one exerts political influence in noncoercive political systems (and to a lesser extent in coercive ones). And it is through framing that political actors shape the texts that influence or prime the agendas and considerations that people think about}

\citeauthor{benkler2006wealth} suggests that a new and more democratic public sphere has been born since the advent of the Internet and social actors who had remained silent so far can now engage in the public dialogue. After stating that to understand \emph{how} frames of meaning are shaped and \emph{by whom} is an important dimension of freedom in contemporary societies, \citeauthor{benkler2006wealth} affirms that \blockcquote[275]{benkler2006wealth}[.]{\textins*{t}he networked information economy makes it possible to reshape both the \enquote{who} and the \enquote{how} of cultural production \textelp{}. It adds to the centralized, market-oriented production system a new framework of radically decentralized individual and cooperative nonmarket production}

\citeauthor{graeff2014battle} apply the concept of \emph{networked framing} to analyze the complex dynamic of the new digital public sphere described by \citeauthor{benkler2006wealth}. Based on the Trayvon Martin issue---about the African American who was shot to death in Florida and that stirred up a nationwide debate on racial inequality---, they argue \blockcquote{graeff2014battle}[.]{the exceptional conditions of a national controversy \textelp{} afford ad hoc emergent framing as networked media sources insert their frames into the coverage and meta-coverage}

Like Trayvon Martin in the United States, the \emph{rolezinhos} was a national controversy in Brazil and it represented an opportunity for online commentators and media activists to get traction for their agendas. This research aims to understand how digital newcomers influenced the public debate through inserting new frames into it.

\section{Methods}
\label{sec:methods}

This is the first study performed with the Brazilian corpus of Media Cloud, a system for large-scale content analysis developed by the Berkman Center for Internet and Society at Harvard and the MIT Center for Civic Media \autocite{about_media_cloud}. Media Cloud currently captures stories published by approximately 3,000 Brazilian media sources. The overall media set comprises more than 30,000 media sources, most of them in English.

So far, a few studies have already been conducted with the American corpora of Media Cloud, such as \citeauthor{benkler2013social}'s analysis of the online debate on SOPA/PIPA---the hollywood-sponsored anti-piracy legislation \autocite{benkler2013social}---and aforementioned \citeauthor{graeff2014battle}'s research on the media coverage of the Trayvon Martin case \autocite{graeff2014battle}. With other corpora, it is worth mentioning a study on the Russian blogosphere by \citeauthor{etling2014blogs} \autocite{etling2014blogs}.

\subsection{Gathering a corpus}

For this study, I downloaded all items in the Media Cloud database related to the \emph{rolezinhos} controversy published between December, \nth{7} 2013 and February, \nth{23} 2014. Initially the corpus for this research had 4,779 articles. After eliminating repeated records and unrelated stories, that number fell to 4,003 articles. Figure~\ref{weekly_distribution} shows the weekly distribution of those texts in the period.

\begin{figure}
\begin{center}
\begin{asy}
bar_graph(numberOfStories);
\end{asy}
\end{center}
\caption{Weekly distribution of texts in the corpus.\label{weekly_distribution}}
\end{figure}

Then it was necessary to find a way to estimate the relative influence of each one of the articles in the corpus. The SOPA/PIPA and the Trayvon Martin studies used a tool called Controversy Mapper \autocite{about_controversy_mapper} to estimate that information. Controversy Mapper represents the corpus as a graph. Each edge in the graph is a link from one article to another. Presumably, the relevance of an article will be directly proportional to the number of inlinks the article has. However, the articles in the corpus of this study showed a rather low number of mutual connections which precluded the use of Controversy Mapper.

Presumably, this may reflect on differences in Brazilian and American blogospheres. It is common to stumble across texts in Brazilian blogosphere that reproduce other websites' articles without adding links to the sources. There is a clear reluctance on the part of traditional media outlets to link competitors and even non-profit organizations. At the same time, tech-savvy activists---unwilling to increase the traffic and pagerank of websites they criticize---are using URL shorteners that redirect to a copy of the original page \autocite{encurtadorurl}. On the other hand, the SOPA/PIPA controversy might be a particularly link-rich corpus due to its tech focus.

For this reason, it was necessary to find another estimation for the relative importance of each article. The sum of comments, likes and shares that a link received in Facebook is a readily available information via the Facebook Graph API \autocite{facebook_graph_api} and seems a good representation of influence due to the social network widespread usage. Currently, Facebook has 76 million users in Brazil, around 38\% of the overall population \autocite{veja_facebook_brasil}. About 47 million of them access Facebook daily. It is the most important social network with a 97.8\% share of the time spent by Brazilians on this type of website \autocite{brasileconomico_facebook_brasil}. For convenience, in the context of this research, the term \emph{total count} will be used as a shorthand for the sum of shares, likes, and comments in a Facebook link and is interpreted as a consolidated measure of the attention received by that URL in the social network.

Since this is the first study performed with the Portuguese sources of Media Cloud, it was important to develop a method to verify if there were blind spots in the media list that feeds the tool. The goal was to ensure the corpus contained all relevant stories. In order to accomplish this validation, I downloaded all the search results for queries with the \emph{rolezinho} keyword in Google for each day in the date range of the controversy. This search generated a list of 12,552 URLs. Around 662 stories were already in the Media Cloud database and 155 of them were duplicated records.

Naturally, it was not convenient to add all of them to the corpus because most of the results were simply unrelated noise or irrelevant comments on the issue. In order to determine the relevant URLs I used again the information from Facebook. The stories that received more attention in the social network (meaning having a total count greater than or equal to 100) were validated manually and, if deemed significant, incorporated to the corpus. After processing the Google data, 520 new stories were included to the corpus. The distribution of those texts in the analyzed time frame is shown in Figure~\ref{google_distribution}.

The whole corpus was eventually comprised by 4,523 stories, about 88.5\% of it coming from Media Cloud and 11.5\% from Google search engine.

\begin{figure}
\begin{center}
\begin{asy}
int[] yValue = {8, 19, 20, 11, 11, 30, 205, 131, 61, 23, 6, 1, 0};
bar_graph(yValue);
\end{asy}
\end{center}
\caption{Weekly distribution of texts from Google.\label{google_distribution}}
\end{figure}

\subsection{Finding frames}

In his comparison between framing and agenda-setting, \citeauthor{maher2001framing} observes that \blockcquote[][\pno~83, emphasis added]{maher2001framing}[.]{\textins*{h}istorically, framing and agenda setting have had opposite trajectories. Agenda-setting began with valuable approaches to measurement, but lacked theoretical depth. By contrast, framing began with roots deep in cognitive psychology, but it has proved to be an \emph{elusive concept to measure}}

In fact, a common methodological problem in the content analysis of media frames is that \blockcquote[503]{vanGorp01122005}{it is extremely difficult to neutralize the impact of the researcher} because \blockcquote{matthes2008content}[.]{a frame is a quite abstract variable} As a result, the identification of media frames is often plagued by some concerns regarding validity and reliability. \blockcquote{matthes2008content}[.]{Some
approaches try to capture latent or cultural meanings of a text, which can be problematic in terms of reliability. Other approaches provide sharp and reliable measures but may fall short in terms of validity}

To answer to those concerns, this study applied a combined approach of manual and computer-assisted methods, as proposed by \citeauthor{matthes2008content}: \blockcquote{matthes2008content}[.]{\textins*{W}e understand a frame as a certain pattern in a given text that is composed of several elements. \textelp{} Rather than directly coding the whole frame, we suggest splitting up the frame into its separate elements, which can quite easily be coded in a content analysis. After this, a cluster analysis of those elements should reveal the frame \autocite{Kohring01042002}. That means when some elements group together systematically in a specific way, they form a pattern that can be identified across several texts in a sample. We call these patterns frames}

Even with the advances in natural language processing, humans greatly outperform machines in understanding texts and searching for components of frames. At the same time, computers usually surpass the human capacity for finding patterns in a wealth of data. The method described here strives to combine the best of both worlds.

The first challenge is to determine the frame elements that should be coded in the content analysis. \citeauthor{entman1993framing}'s definition of frame provides operational criteria to specify those elements. According to him, frames are usually constituted by a \emph{problem definition}, a \emph{causal interpretation}, a \emph{moral evaluation}, and a \emph{treatment recommendation} \autocite{entman1993framing}. Variables that describe all those four dimensions would offer a good---and, to some extent, complete---picture of the frame.

Ideally, the values taken by those variables should be uncontroversially apparent in the articles---in order to diminish the researcher bias---and be convertible to a numeric representation---a convenient input for a cluster analysis algorithm. Accordingly, in this study, the articles were coded based on multiple-choice or yes-no variables that cover the four dimensions of \citeauthor{entman1993framing}'s definition of frame.

Often, the core component in a \emph{problem definition} (\emph{what is the problem?}) is the most important actor of the frame (\emph{who is the problem?}) \autocite[266]{matthes2008content}. In the \emph{rolezinhos} issue, the obvious response is \enquote{the youth}. At the same time, many articles redeem the youth and focus on castigating the whole of society, the malls, or the State. There are also a few texts that opt to take a critical stand on all actors. In the present research, the \emph{problem definition} is summarized in two yes-no variables: Are the youth the problem? Or are the State and/or society?

The reason for gathering all actors---except the participants in \emph{rolezinhos}---under the umbrella term \enquote{State and/or society} is that those actors will be differentiated below, when the \emph{moral judgment} elements are described. In the \emph{problem definition}, a broader question is on the table: are the youth the problem or the victims in the \emph{rolezinhos} issue? Naturally, \enquote{neither} or \enquote{both} are valid answers too.

The \emph{rolezinhos} controversy certainly admits several \emph{causal interpretations} based on economic, social, political, or cultural considerations. The diminishing of poverty in the last decades, for instance, constitutes a valid economic explanation. On the other hand, a description of the values and symbols of \emph{funk ostentação} reveals the cultural underpinnings of those events. Sometimes such interpretations are mutually exclusive but, more frequently, they appear in the same text as a mosaic of arguments.

Every time a new \emph{causal interpretation} was found, I added it to the coding protocol. At the end, if two interpretations were redundant, I usually merged them under a consolidating and more broad label.

If an article explicitly supported a specific interpretation, the corresponding variable was marked with \enquote{Yes} for that article. Accordingly, an explicit rejection of that interpretation yielded a \enquote{No} for the corresponding field. If there was no mention at all, the variable was left in blank. There is a theoretical reason for differentiating explicit negation and mere silence on a causal interpretation. A frame that denies an interpretation is farther from another frame that supports it than a third frame that does not convey any opinion about it. Such distinctions must be acknowledged.

For the \emph{moral evaluation}, the content analysis assessed the perceptions conveyed by each text about the actors involved in the \emph{rolezinhos} issue: participants, courts, malls, city hall, federal government, state government, society at large, progressives, conservatives, media etc. A five-item scale was used in the assessment---with the options \enquote{very favorable (opinion)}, \enquote{favorable}, \enquote{neutral}, \enquote{unfavorable}, and \enquote{very unfavorable}.

Four yes-no variables described the \emph{treatment recommendation} component of the frame analysis. Two of them comprised options related to the attempt to avoid or divert the \emph{rolezinhos}. The first was the flat prohibition of the events in malls through court injunctions and the threat of fines or detention. The second variable represented the attempt of negotiating with the promoters of \emph{rolezinhos} to move the events to public parks or other venues. The social movements usually proposed the solution described in the third variable: the escalation of tensions with more \emph{rolezinhos} and protests. Finally, the fourth variable described the invitation to \enquote{dialogue and reflection}, the premise being that the violent reaction against the \emph{rolezinhos} was a consequence of ignorance about the youth culture in the suburbs.

Table~\ref{content_analysis} shows the frame elements as defined by \citeauthor{entman1993framing} and the corresponding variables in our content analysis.

\begin{table}[ht]
\centering
\resizebox{\textwidth}{!}{%
\begin{tabular}{@{}lll@{}}
\toprule
Frame element            & \multicolumn{2}{l}{Variables}                        \\ \midrule
problem definition       & main actor:    & youth                               \\
                         &                & state and society                   \\
causal interpretation    & economic:      & new middle class                    \\
                         & social:        & social criticism                    \\
                         &                & lack of civility                    \\
                         &                & social or racial prejudice          \\
                         & cultural:      & funk ostentação and social networks \\
                         &                & few options for recreation          \\
moral evaluation         & direct actors: & youth                               \\
                         &                & malls                               \\
                         &                & police                              \\
                         & government:    & city level                          \\
                         &                & state level                         \\
                         &                & federal level                       \\
                         &                & courts                              \\
                         & society:       & media                               \\
                         &                & progressives                        \\
                         &                & conservatives                       \\
treatment recommendation & against:       & fines and detentions               \\
                         &                & rolezinhos in public parks          \\
                         & pro:           & more rolezinhos and protests        \\
                         &                & dialogue and reflection             \\ \bottomrule
\end{tabular}
}
\caption{Variables for Cluster Analysis\label{content_analysis}}
\end{table}

Obviously, the choice of frame elements introduces the possibility of researcher bias. \citeauthor{matthes2008content} recognize this problem: \blockcquote{matthes2008content}[.]{The problem reliability in frame analysis is not completely resolved but is shifted to the content analytical assessment of single frame elements. However, the more manifest a certain variable is, the higher is its reliability. Therefore, single frame elements achieve a higher reliability in comparison to abstract, holistic frames. \textelp{} \textins*{C}oding holistic frames is one of the major threats to reliability in frame analysis. Another crucial advantage of this method is that coders do not know which frame they are currently coding, as they are not coding frames as single units. Thus, the impact of coder schemata or coding expectations is weaker. Moreover, new emerging frames can be easily detected}

Two coders were responsible for the analysis of the 60 most popular articles according to the Facebook \emph{total count} ranking. The identification of frames was based on them. As I am looking for influential frames, this focus on the most liked, commented, and shared links is reasonable. Attention in media is widely believed to follow a Pareto distribution \autocite{miotto2014predictability}, with a small number of articles responsible for a disproportionate share of attention. Our corpus resembles this pattern as well: those 60 articles represent 1.3\% of the corpus, but 55\% of likes, comments, and shares.

After the content analysis, the codes were converted to numbers in order to be used as an input for the cluster algorithm. Tables~\ref{table_yesno} and \ref{table_fivescale} describe the criteria used for the numeric translation. Each one of the frame dimensions was multiplied by a different factor. The \emph{problem definition} was tripled because it has a high importance in the frame analysis but it is comprised of only two variables in our coding schema. The \emph{treatment recommendation} remained the same because most of the analyzed texts did not mention that dimension and hence it was not a good discriminating criteria. The \emph{causal interpretation} and the \emph{moral evaluation} was doubled, using therefore a middle factor of $2$. Then, the articles were grouped using a hierarchical cluster analysis (Ward method), the same technique successfully applied in previous studies by \citeauthor{matthes2008content} \autocite{matthes2008content,Kohring01042002}.

\begin{table}
\centering
\begin{tabular}{@{}llll@{}}
\toprule
\textit{Code}  & No & Blank & Yes \\ \midrule
\textit{Value} & -1 & 0     & 1   \\ \bottomrule
\end{tabular}
\caption{Translation of yes-no questions.\label{table_yesno}}
\end{table}

\begin{table}
\centering
\begin{tabular}{@{}llllll@{}}
\toprule
\textit{Code}  & Very unfavorable       & Unfavorable            & Neutral               & Favorable             & Very favorable        \\ \midrule
\textit{Value} & \multicolumn{1}{c}{$-2$} & \multicolumn{1}{c}{$-1$} & \multicolumn{1}{c}{$0$} & \multicolumn{1}{c}{$1$} & \multicolumn{1}{c}{$2$} \\ \bottomrule
\end{tabular}
\caption{Translation of the five-scale responses.\label{table_fivescale}}
\end{table}

However, there are two significant innovations. First, in \citeauthor{matthes2008content}'s research, the numeric conversion of the content analysis engendered only binary variables. As the previous paragraph shows, this is not the case here.

Also, those authors determined the number of frames based on the so-called elbow criterion---similar to a scree test in exploratory factor analysis. This research employed the approximately unbiased probability values (\emph{p}-values) method described by \citeauthor{suzuki2004application}---a more specific strategy for assessing how strong a cluster is supported by the data \autocite{suzuki2004application, Suzuki15062006}.

Interestingly, \citeauthor{suzuki2004application} developed their algorithm in the context of biochemical research. They used hierarchical cluster analysis to perform DNA comparison. In fact, this method is commonly used to draw phylogenetic trees: a graphic representation of the evolutionary relationships among different species. For instance, if a group of butterflies belong to the same biological genus or family, they are placed together in the same branch of the phylogenetic tree.

It is possible to draw a parallel between hierarchical cluster analysis applications both in biological and communication studies. The frame elements defined above are like pieces of the frame DNA. The algorithm processes them and outputs a tree that places together in the same branch those articles that belong to the same \enquote{genus or family of arguments}: such subtrees are the frames I am looking for.

For this study, the outcome of the hierarchical cluster analysis is a binary tree where the leaves are the articles and the internal nodes are the \emph{p}-values. A \emph{p}-value is a percentage that indicates the probability of that node being the root of a meaningful cluster. Subtrees with higher \emph{p}-values are more likely of being real world frames. With the clue given by the \emph{p}-values, the next step is to read the texts in each cluster to validate the frames.

\subsection{Evaluation of the results}

Entman posits that the frames \blockcquote{entman1993framing}[.]{are manifested by the presence or absence of certain keywords} \blockcquote{ghaziani2005keywords}[.]{\textins*{A} keyword can be thought of as a semantic isotope: a cultural tag, tracer, or dye that tracks changes in meaning deployed by diverse social actors during periods of change} Accordingly, many researchers decided to map the frames using keywords \cite{Miller01121997, miller1998framing, Fiss01022005}.

In this research, the next step after validating the clusters was to find the keywords that manifest the frames. Those keywords served as parameters for natural language processing (NLP) strategies that tracked and compared the evolution of frames over time in the whole corpus.

Since \emph{rolezinhos} always happened in the weekends, most of the analysis was performed in a weekly basis (with Monday as the first weekday). A ranking of the articles was established for each week based on the Facebook total count for the corresponding URL. Based on that ranking, it was possible to identify who were the voices setting the agenda each week.

Besides the most influential voices, I looked for the genesis of each frame and tried to determine the first voices to propose new frames. Interviews with strategic actors were also performed in order to validate the findings and include additional background information and quotes.