\chapter{Critical analysis}

\section{Comparisons}

In the sample of the 60 most popular articles in the corpus, the \emph{apartheid frame} does significantly better than the other two competing frames, both in the number of articles and in the estimated attention based on Facebook \emph{total count}.

However, it is interesting to compare the differences among number of articles and Facebook \emph{total count} (Figure~\ref{artigos_atencao_cluster}). The share of the \emph{apartheid frame} when it comes to number of articles is 10\% greater than its \emph{total count}. At the same time, with only 5\% of the articles in the corpus, the \emph{middle ground frame} manages to receive 13\% of the overall attention. The \emph{arrastão frame} remains with a similar ratio.

\begin{figure}
\centering
\begin{subfigure}[b]{0.3\textwidth}
\centering
\input{cap4-1b.pdf_tex}
\caption{Number of articles}\label{subfig:p1}
\end{subfigure}
\begin{subfigure}[b]{0.3\textwidth}
\centering
\input{cap4-1a.pdf_tex}
\caption{Total count}\label{subfig:p2}
\end{subfigure}
\begin{subfigure}[t]{0.3\textwidth}
\centering
\input{cap4-1legenda.pdf_tex}
\end{subfigure}
\caption{Share of (\subref{subfig:p1}) number of articles and (\subref{subfig:p2}) Facebook \emph{total count} by each frame.}\label{artigos_atencao_cluster}
\end{figure}

Part of the \emph{apartheid frame}'s \emph{inefficiency} in converting number of articles in Facebook audience might be related to its poor use of video, a highly spreadable medium in social networks.

\subsection*{Video and text}

Figure~\ref{fig:cap4-2a} suggests that Brazilian progressives---the main sponsors of the \emph{apartheid frame}---heavily relied in the written word. Only one fifth of its media was video, against one third in the other frames. Further, \emph{apartheid}'s movies were not as successful. The average \emph{total count} for them was 14,408. By comparison, the same amount for \emph{arrastão} and \emph{middle ground} were 34,813 and 86,228 respectively (although it is worth remembering that the \emph{middle ground frame} is comprised by only three articles).

\begin{figure}
\begin{subfigure}[b]{0.45\textwidth}
\caption{Number of articles}\label{fig:cap4-2a}
\input{cap4-2a.pdf_tex}
\end{subfigure}
\begin{subfigure}[b]{0.45\textwidth}
\caption{Total count}\label{fig:cap4-2b}
\input{cap4-2b.pdf_tex}
\end{subfigure}
\caption{Proportion of text and video in each frame by (\subref{fig:cap4-2a}) number of articles and (\subref{fig:cap4-2b}) Facebook \emph{total count}.}\label{fig:cap4-2}
\end{figure} 

Around 40\% of the Facebook activity in the \emph{apartheid frame} comes from the \emph{progressive commentators subcluster} (Figure~\ref{fig:cap4-3a}) that does not have a single video. The second most popular piece in that subcluster is \citeauthor{brum_vandalos}'s \blockcquote{brum_vandalos}{The new Brazilian \enquote{thugs}}, a text with startling 49,000 characters. On the other hand, the main subcluster in the \emph{arrastão frame} is called \emph{Youtube subcluster} with 46\% of the attention share (Figure~\ref{fig:cap4-3b}). As a result, while the video share of \emph{apartheid frame}'s \emph{total count} is around 16\%, the same variable for the other frames amounts to roughly 60\% (Figure~\ref{fig:cap4-2b}).

\begin{figure}
\begin{subfigure}[b]{0.5\textwidth}
\centering
\caption{Apartheid}\label{fig:cap4-3a}
\input{apartheid.pdf_tex}
\end{subfigure}
\begin{subfigure}[b]{0.5\textwidth}
\centering
\caption{Arrastão}\label{fig:cap4-3b}
\input{arrastao.pdf_tex}
\end{subfigure}
\caption{Share of Facebook \emph{total count} by each subcluster in (\subref{fig:cap4-3a}) \emph{apartheid frame} and (\subref{fig:cap4-3b}) \emph{arrastão frame}.}\label{fig:cap4-3}
\end{figure} 

\emph{Apartheid frame} supporters usually favored a stern approach with long and thoughtful texts. There is only one comic video in their cluster: a mashup based on a well-known scene of \emph{Downfall} \autocite{hitler_rolezinhos}. Nazi officials explain to Hitler the phenomenon of the \emph{rolezinhos} and a disturbed führer proposes a violent reaction. A hesitant officer observes that \enquote{poor people should have the right of having fun.} There is an implicit analogy between Hitler and the ruling authorities in the places where the \emph{rolezinhos} happened---specifically, the governor of São Paulo, Geraldo Alckmin. On the other hand, the \emph{arrastão frame} counted on at least four videos with comic content. Three of them---already described in \autoref{sub:youtube} \nameref{sub:youtube} on page \pageref{sub:youtube}---are a very caustic attack on the participants of \emph{rolezinhos}.

\subsection*{Ebb and flow}

Nonetheless, the \emph{apartheid frame}'s preeminence is indisputable. It represents almost 50\% of all Facebook activity in the sample. An analysis of the aforementioned \emph{progressive commentators subcluster} shows that, among its five most influential articles, four of them were published in December (Figure~\ref{fig:cap4-4}), before the media storm in the third week of January. That is an indication of their importance in framing the \emph{rolezinhos} issue in the onset of the debate.

\begin{figure}
\centering
\input{progressives.pdf_tex}
\caption{Facebook \emph{total count} per article in \emph{progressive commentators subcluster}.}\label{fig:cap4-4}
\end{figure} 

The same chart also suggests the predominance of traditional media, influential journalists, or well-established sources in the debate. There is only one article that comes from an alternative source---the blog post by activist Maria Frô \autocite{maria_fro}.

It is worthwhile to see how the three frames interacted with each other during the weeks of the controversy. The scatter chart in Figure~\ref{fig:cap4-5} offers an interesting visualization that relates the impact of each article with the chronological evolution of the controversy.

\begin{figure}
\centering
\input{scatter.pdf_tex}
\caption{Articles distributed according to publish date and Facebook \emph{total count}.}\label{fig:cap4-5}
\end{figure} 

It seems clear that, during the first two weeks, the \emph{arrastão frame} was the only voice framing the debate. Some articles---namely, the ones about the first \emph{rolezinho} in \emph{Shopping Metrô Itaquera}---had a remarkable spread even if compared to the most influential articles in subsequent weeks. In the third week, the \emph{apartheid frame} came to challenge that hegemony. In fact, during late-December and early-January, it was an uncontested perspective.

After the incidents on January \nth{11}---rubber bullets in \emph{Shopping Metrô Itaquera} and court injunctions in other malls---, a real battle of frames began. In the noisy environment of that week, the predominant force continued to be the \emph{apartheid frame}. In the following week, there was a reaction of the \emph{arrastão frame} that lasted for a few days. Finally, the \emph{apartheid frame} became the most common interpretation floating around on social media. However, the controversy was already waning to become a side topic in other issues.

It is possible to recognize the ebb and flow of both frames, although it seems the \emph{apartheid frame} ended as the prevailing toolkit to interpret the \emph{rolezinhos} in Brazilian media. A much more complex question---and outside the scope of this research---is the concrete impact of such media consensus in public opinion. Taken at face value, aforementioned \emph{Folha de S. Paulo}'s survey on January \nth{23} \autocite{folha_datafolha} discourages any proud boast by \emph{apartheid frame} supporters.

The \emph{middle ground frame} proved to be a very successful frame during the peak of the debate. When interviewed for this research, Beguoci attributed his article's success to the perfect timing of its publication \autocite{interview_beguoci}. There was a craving for well-informed interpretations among media, government, and citizens. However, the \emph{middle ground frame} did not succeed in winning strong support for its view as the following weeks demonstrated.

\subsection*{Government and media}

Nothing is better to gather momentum for a media storm than a truculent response from the public security forces. That has been an oft-ignored lesson in recent Brazilian history. In June 2013, a movement that started with a boring protest against a 10-cent increase in bus fares became a nation-wide turmoil after São Paulo police decided to use tear gas and rubber bullets to curb the demonstrations.

Similarly, the data in this research supports the hypothesis that the surge in media coverage followed closely behind the use of aggressive means of coercion on January \nth{11} and added legitimacy to the \emph{apartheid frame} discourse that the State is at the service of social and racial elites.

In fact, minister Gilberto Carvalho's was not alone when he said that the police reaction is equivalent to \blockcquote{folha_carvalho_fogo}[.]{throwing gasoline in a bonfire} Elio Gaspari, an influential columnist in \emph{Folha de S. Paulo}, suggested: \blockcquote{eliogaspari}[.]{The best starting point for dealing with \emph{rolezinhos} is to ignore those solutions that exacerbate the problem} In fact, after the severe media backlash, the authorities in charge of the problem started to reach out the youth and try more conciliatory approaches.

Significantly, \emph{rolezinhos} have become a term on its own in the Brazilian social lexicon and, unlike \emph{arrastão} and \emph{apartheid}, with a fairly positive overtone. Different groups have used the word to promote various activities: a \emph{rolezinho} for blood donation, a Christian \emph{rolezinho}, a legislative \emph{rolezinho} (to approve new bills) etc.

\subsection*{Race}

Race is a thorny issue in Brazil. In 2012, when the government passed a bill that instituted racial quotas in all federal universities, for instance, some political commentators criticized what they saw as an attempt to import a social conflict (from the United States) extraneous to the Brazilian context. The premise was that, after slavery was outlawed in the \nth{19} Century, the relationship among different racial groups has been peaceful and friendly. If there is an apartheid, it is not racial, but social. Accordingly, some of those commentators were willing to support quotas based on socioeconomic criteria, but staunchly opposed to racial ones.

Eventually, racial quotas were approved and have been implemented all over Brazil. Nevertheless, now and then the discussion over racism in Brazil surfaces, like in the \emph{rolezinhos} issue. The questions are usually the same: is there an apartheid? If so, is it a racial or merely economic one?

In his interview \autocite{interview_godinho}, Godinho said that the idea of using the word \emph{apartheid} to describe a \emph{rolezinho} came from black organizations in Brazil. They had been using that frame to describe a similar incident in \emph{Shopping Vitória} (see \autoref{sec:progressives} \nameref{sec:progressives} on page \pageref{sec:progressives}). The same organizations were also responsible for promoting demonstrations in affluent shopping malls, after the incidents on January \nth{11}, further promoting the \emph{apartheid frame}.

As expected, some critics explicitly challenged the racial component of the frame (see \autoref{sec:neocons} \nameref{sec:neocons} on page \pageref{sec:neocons}). \citeauthor{constantino_comportamental}---in an article significantly titled \blockcquote{constantino_comportamental}{The apartheid in shopping malls is a behavioral one}---compared pictures of a white woman being thoroughly searched at a mall's door and a black family having lunch at another mall without suffering any embarrassment from the nearby police. His point was that there is no racism in the reaction against \emph{rolezinhos}. Other 

Nevertheless, the fact that virtually every commentator on \emph{rolezinhos} was obliged to take a stand on the racial problem---no matter if positive or negative---represents a considerable victory for the black organizations in Brazil and a sign that their framing efforts were successful in determining the terms of debate. Figure~\ref{chart_apartheid} is a graphic representation of that success for it suggests how the \emph{apartheid frame} gradually deallocated the more bluntly racist elements of the \emph{arrastão frame}---to start with, the use of the term \emph{arrastão} itself.

\section{Innovations, limits, and future directions}

In addition to a better comprehension of the \emph{rolezinhos} issue and the Brazilian media ecosystem, this research provided three methodological innovations that are worth mentioning.

\begin{enumerate}
\item It is the first use of the Brazilian corpora of Media Cloud and, as such, a pilot study to validate it. So far, the results are encouraging. 
About 88.5\% of the analyzed texts came from Media Cloud and only 11.5\% from complementary sources, a fairly good ratio. One of the most important outcomes of this research is a road map to adapt Media Cloud for other languages.
\item Facebook data proved to be a consistent estimation for the audience and relative relevance of URLs in the Media Cloud corpus. The use of like, share, comment, and total count is a promising approach for future research.
\item As a means of achieving a higher level of validity and reliability in framing studies, hierarchical cluster analysis seemed a consistent method. This research elaborated the original approach suggested by \citeauthor{matthes2008content} \autocite{matthes2008content}.
\end{enumerate}

Among the limits of this research strategy, it is possible to recognize that one frame frequently shares some points of view with other frames. A possible representation of those shared beliefs is presented on Table~\ref{tab_comparison}.

\begin{table}
\centering
\begin{tabular}{@{}p{3cm}p{2cm}p{2cm}p{2cm}p{2cm}p{2cm}@{}}
\toprule
Frame or actor       & There were thefts & Successful social policies & There were prejudices & Police overreacted & Rolezinhos are lawful \\
\midrule
Arrastão              & yes                                  & no                                                           & no                                                       & no                                       & no                          \\
Apartheid             & no                                   & yes                                                          & yes                                                      & yes                                      & yes                         \\
Middle Ground         & --                                   & --                                                           & yes                                                      & yes                                      & no                          \\
Mall's case           & no                                   & --                                                           & no                                                       & no                                       & no                          \\
PT's case & no                                   & yes                                                          & --                                                       & yes                                      & --                          \\
\bottomrule
\end{tabular}
\caption{Comparison of opinions associated to frames and some actors.}
\label{tab_comparison}
\end{table}

Therefore, it is not trivial to disentangle the different frames and, in fact, there were cases when an article presents multiple---even contradictory---frames, for example, when a story wrote with the \emph{arrastão} perspective also quotes the storekeepers on the need of police action. A future improvement might be to code groups of paragraphs instead of whole articles.

A reasonable criticism to the method of using keywords in the computational analysis to identify stories belonging to a frame is that a keyword can also be mentioned in order to discredit a frame. An article that challenges the \emph{arrastão} frame, for instance, will normally use the word \emph{arrastão}, perhaps in the title \autocite{badernamidiatica}, but it will try to enumerate the possible flaws of the \emph{arrastão} argument.

Based on George Lakoff's reflections on contemporary public discourse \autocite{lakoff2008don}, it is possible to argue that to engage in a dialogue, even a controversial one, with the \emph{arrastão} or the \emph{apartheid} frames is, at least implicitly and pragmatically, equivalent to accepting those frames and contributing to their spread in the public arena.

Another limitation is that Facebook data gives a good estimation of audience but without any hint of how knowledgeable is the readership. In fact, there is a considerable difference if an article influences 300 13-year-old boys or 300 mainstream media reporters. The latter has arguably more tools and skills to spread the word.

The content analysis was performed with 60 articles. Although it is a small sample, it represents 55\% of all Facebook activity in the corpus. This study is based on the assumption that, with a good audience estimation, it is possible to focus the content analysis on the articles with higher impact in public opinion. At the end of the day, we usually do not need an exhaustive inventory of frames, but a reliable depiction of the framing strategies that really mattered for the public discourse. Nonetheless, it might be interesting to increase the sample in future research. Given the uneven distribution of attention among articles, a sample with the 300 most popular URLs would represent 86\% of the Facebook activity in the \emph{rolezinho} corpus.

It would also be convenient to consider each one of the Facebook variables---like, share, and comment count---separately. In fact, their social meaning is different. To press a \enquote{like} button usually means \enquote{support} to an idea, a cause, or some piece of news. The number of comments is a good proxy for the buzz caused by an article. The number of shares has a more ambiguous meaning: it normally indicates support but Facebook users also share articles that they particularly dislike, usually with a message stating their opinion. It is also true that a user can \enquote{like} a link because it was posted with a deprecating commentary, in which case the like to that post would mean a dislike to the link itself. The meaning can also depend on context. A flood of comments in a politically hot issue might mean something different than the same amount of comments in a funny video. Therefore, in order to use those individual variables, it would be necessary to propose a consistent model to accurately interpret and relate the numbers.

Finally, in future studies, the hierarchical cluster analysis could also be performed for each individual period of the controversy. Thus short-living frames or subclusters could be identified more easily. However, a bigger corpus is advisable. In the \emph{rolezinhos} issue, some weeks had only a few dozens of articles, a limited input for the clustering algorithm.