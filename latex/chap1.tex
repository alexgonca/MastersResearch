\chapter{Introduction}

The aim of this study is to examine how the Brazilian media and blogosphere covered the \emph{rolezinhos} issue. It is not trivial to find an accurate English translation for the term \emph{rolezinho}. Literally, it is a \enquote{little walk} or a \enquote{brief stroll.} Since December 2013, the word has meant something very concrete in Brazilian media parlance: flashmobs organized by youth---mostly low income ones---in shopping malls. The meetings usually gathered from dozens to thousands of people and frequently triggered a harsh reaction by the local police.

This research does not intend to analyze the economic causes and sociopolitical consequences of the \emph{rolezinhos}, but rather the discourses and dialogues created around this phenomenon in the Brazilian media. It aims to identify the main frames used to talk about \emph{rolezinhos}, their origin and how some frames managed to achieve certain prominence in the public arena.

Some features inherent to the subject of \emph{rolezinhos} make it especially suitable for agenda-setting and media framing analysis.
\begin{enumerate}
\item The \emph{rolezinhos} issue is a neatly time-delimited phenomenon. The first gathering that mobilized media attention took place on December \nth{7}, 2013, \autocite{folha_arrastao_itaquera, agora_arrastao_itaquera, band_arrastao_itaquera, g1_arrastao_itaquera, terra_arrastao_itaquera, estado_arrastao_itaquera, veja_arrastao_itaquera}. In the third week of February, the controversy had already cooled down and was only receiving marginal media coverage.
\item The controversy presents an obvious keyword (\emph{rolezinho}) that has no concurrent meanings. Therefore, it is easier to use computational methods to assemble the corpus and perform a significant part of the analysis.
\item The \emph{rolezinhos} phenomenon is a complex one and a single interpretation is not obvious. In addition, since the young promoters of the \emph{rolezinhos} usually do not provide sophisticated analyzes about their own motivations and intentions, experts from various ideological backgrounds are summoned to quench the public's thirst for explanations. The media debate becomes then a war of frames and each actor uses the controversy to push forward with his own agenda of social change or political reform.
\end{enumerate}

In the following sections, a narrative of the phenomenon will be presented in a timeline that spans for about two months as well as a brief explanation of the historical, social, and political context of the \emph{rolezinhos}. Then, the theoretical framework that guided this research will be outlined. The methods for frame identification will be thoroughly described and applied to understand the \emph{rolezinhos} issue.

Finally, a critical review will reflect on the main contributions of this research: namely, the proposition of a new method to perform frame analysis---based on large scale content analysis, social network data, and cluster analysis---and the description of underlying frames that pervade racial and class conflicts in Brazil.

\section{Timeline}

From the beginning, it is convenient to have a general idea of how the \emph{rolezinhos} issue evolved. The following is a brief summary of the main facts in chronological order.

\subsection*{The first rolezinho (12/07/2013--12/13/2013)}

On Saturday, December \nth{7}, 2013, around 6,000 young people gathered at 5~p.m. at \emph{Shopping Metrô Itaquera}, a mall in the East Side of the city of São Paulo, for the first \emph{rolezinho} that drew media attention. Youngsters from the outskirts of São Paulo organized the event through Facebook. The event’s description asserted, \blockcquote{facebook_2013-12-07}{nobody \textins{at the meeting} is famous} (implying that it was not a fan gathering). The aim was just \enquote{to meet friends, mess around, kiss, and take photos.} In the title, the \emph{rolezinho} was defined as \enquote{Part 2} because a similar event \autocite{facebook_2013-11-30} had happened at the same place on November \nth{30}, Saturday, at 3~p.m., but with few people and no press coverage or police reaction.

Youtube videos show the gathering in the parking lot \autocite{youtube_primeiro_rolezinho_outside} and inside the mall \autocite{youtube_primeiro_rolezinho_inside}. The security guards were unable to disperse the crowd and the mall called the police. Shopkeepers and customers got scared with the noisy crowd and some of them shut themselves in the stores \autocite{agora_arrastao_itaquera}. Two people were detained: one was carrying a toy gun---a replica of a .38 revolver---and the other was already wanted by the police for a previous offense of unlawful possession of a firearm \autocite{estado_arrastao_itaquera}. On Twitter \autocite{twitter_arrastao_itaquera} and Facebook \autocite{facebook_nota_metro_itaquera}, consumers mentioned thefts at the mall and in the vicinity, but no crime was reported at the police stations so it is difficult to confirm if instances of theft occurred. That day, \emph{Shopping Metrô Itaquera} decided to close an hour and a half earlier, at 8:30~p.m.

Between December \nth{8} and December \nth{13}, the news media started to follow the \emph{rolezinhos} scheduled on Facebook for the following weeks \autocite{agora_novas_invasoes,estado_novas_invasoes,veja_novas_invasoes,r7_novas_invasoes,estado_novas_invasoes_2}. Two planned events for the weekend of the \nth{14} and the \nth{15} were particularly popular on the Web: the \emph{rolezinhos} at \emph{Shopping Aricanduva} and at \emph{Internacional Shopping Guarulhos}. The first---with 9,000 confirmed Facebook users---was canceled due to the organizer's fears of turmoil and of the police reaction \autocite{cancelamento_shopping_aricanduva}. Eventually, only the \emph{rolezinho} at \emph{Internacional Shopping Guarulhos} occurred on Saturday, December \nth{14}.

\subsection*{The initial spread (12/14/2013--12/22/2013)}

About 2,500 people attended the \emph{rolezinho} at the \emph{Internacional Shopping Guarulhos} \autocite{g1_guarulhos}, a smaller crowd than the 6,000 in the previous week. However, the police reaction was harsher and 23 people were taken to the police station \autocite{estado_guarulhos,folha_guarulhos}, though they were released during the early hours of the following day \autocite{folha_libertacao_guarulhos}.

One week later, on Saturday, December \nth{21}, there was another \emph{rolezinho} at \emph{Shopping Campo Limpo}. However, fewer than 200 people gathered there because the presence of the police displaying rubber bullet guns and tear gas discouraged any adventures \autocite{folha_campo_limpo_1}. No one was arrested. Nevertheless, some stores decided to close during the gathering \autocite{folha_campo_limpo_2}.

The next day, Sunday, December \nth{22}, another \emph{rolezinho} happened at \emph{Shopping Interlagos} \autocite{folha_interlagos,g1_interlagos,ig_interlagos}. Neither the police nor the shopping mall gave estimations on the number of participants, but the Facebook invitation to the event had more than 9,000 confirmations \autocite{r7_interlagos}. Although all news pieces mention detentions, the reports are inconsistent. The number of detainees for that single gathering at \emph{Shopping Interlagos} ranges from two \autocite{r7_novas_invasoes} to 25 \autocite{estado_interlagos} young men depending on the news article.

\subsection*{On the verge of the conflict (12/23/2013--01/10/2014)}

From Christmas to January \nth{10}, there was only one \emph{rolezinho} in São Paulo that grabbed media attention. On January \nth{4}, around 400 youths gathered at \emph{Shopping Metrô Tucuruvi}, in the North Side of São Paulo \autocite{folha_tucuruvi_1,jovempan_tucuruvi,g1_tucuruvi}. There were no crime reports but the mall closed three hours early, at 7~p.m. \autocite{folha_tucuruvi_2,estado_tucuruvi}.

During that period, there were the first reports of \emph{rolezinhos} outside the city (Figure~\ref{citiesinsidesaopaulo}) and the state of São Paulo (Figure~\ref{citiesoutsidesaopaulo}).

\begin{figure}
\centering
\input{Brazil.pdf_tex}
\caption{Cities outside São Paulo State that hosted \emph{rolezinhos} during the research period.}
\label{citiesoutsidesaopaulo}
\end{figure}

\begin{figure}
\centering
\input{SaoPauloState.pdf_tex}
\caption{Cities in São Paulo State that hosted \emph{rolezinhos} during the research period. (Only the most relevant cities for the controversy are labeled.)}
\label{citiesinsidesaopaulo}
\end{figure}

In Campinas, a town 70 miles away from São Paulo, there was an attempted \emph{rolezinho} on December \nth{27} \autocite{cbn_campinas} at \emph{Shopping Unimart} but the local police impeded the gathering. On January \nth{3}, however, the youth organized a successful \emph{rolezinho} at the same place with no incidents \autocite{correio_popular_campinas_1}. On the following day, January \nth{4}, there was another event at \emph{Shopping Parque das Bandeiras}, but the security guards shut out any unaccompanied adolescents \autocite{g1_campinas}. On January \nth{10}, a group of 60 adolescents scheduled a meeting at \emph{Shopping Iguatemi}, but they decided to move the \emph{rolezinho} in the face of the mall's unwillingness to host the event. Eventually, they ended up at a hamburger fast food restaurant. Three police cars were parked nearby but the officers did not feel like interfering with the party \autocite{correio_popular_campinas_2}.

In Araraquara, another city in the São Paulo State, security guards expelled a small group of 10 youths from \emph{Shopping Jaraguá} on January \nth{4}. According to the mall, they were interfering with other customers \autocite{araraquara}.

In Maceió, Northeastern of Brazil, 15 youth were kicked out of a shopping mall on December \nth{27} \autocite{maceio} under the suspicion of taking part in a \emph{rolezinho}.

On January \nth{5}, about 200 people gathered at \emph{Shopping Palladium} in the city of Ponta Grossa---Paraná State---in the Brazilian South. The mall called the police and closed earlier than usual \autocite{arede_pontagrossa,gazeta_cascavel_pontagrossa}. In Cascavel, another city of the same state, there was a scheduled meeting on the same day in \emph{Shopping Cascavel JL}, but the police and the rain dampened participation \autocite{cgn_cascavel,gazeta_cascavel_pontagrossa}.

\subsection*{Catching fire (01/11/2014--01/26/2014)}

On Saturday, January \nth{11}, police used tear gas and rubber bullets to break up a \emph{rolezinho} in \emph{Shopping Metrô Itaquera}, the same mall that hosted the first \emph{rolezinho}. Two people were arrested \autocite{band_confronto_itaquera,folha_confronto_itaquera,ig_confronto_itaquera,estado_confronto_itaquera,g1_confronto_itaquera,r7_confronto_itaquera}.

The shopping malls \emph{Campo Limpo} \autocite{cartacapital_liminar_shoppings} and \emph{JK Iguatemi} \autocite{folha_liminar_shoppings,estado_liminar_shoppings,veja_liminar_shoppings,g1_liminar_shoppings,band_liminar_shoppings} obtained court injunctions prohibiting the holding of \emph{rolezinhos}. In practice, the injunctions authorized the malls to deny the access of people considered suspicious at the mall's discretion. In addition, anyone who tried to organize a \emph{rolezinho} was threatened with a fine of 10,000 Brazilian reais (US\$4,500). Other shopping malls adopted the same strategy in the following weeks.

The tougher response from the malls and the police caused many young people to give up on organizing \emph{rolezinhos} in São Paulo, but they brought new actors to the stage: various social movements organized ``protest-\emph{rolezinhos}'' in São Paulo and Rio de Janeiro against racial and social discrimination \autocite{estado_rolezinho_leblon, folha_rolezinho_leblon, veja_rolezinho_passeata}. The Homeless Workers' Movement \autocite{folha_semteto} and the Black Movement \autocite{folha_protesto_jk} were responsible for these demonstrations. In addition, there was a politicization of the debate with the exchange of criticisms between the federal government and the state government of São Paulo on the proper response to the \emph{rolezinhos} issue \autocite{folha_ministra_reacaobrancos,estado_rolezinho_oposicao}. Figure~\ref{saopaulodistrict} helps to situate the key places for the \emph{rolezinho} issue in the city of São Paulo.

The media coverage soared. At the same time, dozens of \emph{rolezinhos} were organized all over the country, especially in the other State capitals \autocite{atarde_rolezinho_salvador, opovo_rolezinho_fortaleza, bhaz_rolezinho_estacao, londrina_rolezinho, campogrande_rolezinho}.

\begin{figure}
\centering
\input{SaoPauloCity.pdf_tex}
\caption{Places that hosted \emph{rolezinhos} in the city of São Paulo during the research period. The colors show the human development index (HDI) for each district in the city \autocite{atlasmunicipal}.}
\label{saopaulodistrict}
\end{figure}

\subsection*{Fading away (01/27/2014--02/09/2014)}

After two weeks, the \emph{rolezinhos} issue began to lose relevance to other topics in the headlines. On February \nth{3}, for example, a young black man, accused of theft, was beaten by an unidentified group of citizens, stripped of his clothing and chained by the neck with a bike lock to a streetlight in Rio de Janeiro \autocite{folha_jovem_acorrentado,g1_jovem_acorrentado}. A picture of him spread on the Internet and motivated an intense debate on discrimination and violence.

Three days later, protests against the World Cup in Brazil ended up in a tragedy. One protester threw a firecracker at the police in Rio de Janeiro, but the explosive eventually detonated near the head of a cameraman who was recording the confrontation \autocite{folha_cinegrafista, g1_cinegrafista}. The cameraman's death caused great consternation in the public opinion.

\subsection*{A tame rolezinho (02/10/2014--02/23/2014)}

The organizers of the first \emph{rolezinhos} agreed to establish a partnership with the city hall of São Paulo \autocite{folha_associacao_prefeitura}. They would hold meetings in city parks, as well as some public cultural centers in poor neighborhoods \autocite{g1_associacao_prefeitura}. The first meeting took place on February \nth{15} at Ibirapuera Park and gathered only 100 people \autocite{folha_rolezinho_ibirapuera}. Malls supported the agreement and said they would be willing to allow \emph{rolezinhos} inside the malls if there was a commitment of gathering no more than 600 people at each meeting \autocite{g1_mp_shoppings}. Sporadic \emph{rolezinhos} received less and less media coverage.

\section{Historical background and context} \label{historical_background}

Even when the \emph{rolezinhos} became a national issue, its epicenter continued to be the Greater São Paulo and its suburban youth culture, especially the dressing and behavior codes associated with the music style known as \emph{funk ostentação} \autocite{epoca_origem_rolezinho}.

In the 1970s, radios and parties popularized American funk in the slums of Rio de Janeiro \autocite{cunha_2013_tapatrao}. From the late 1980s on, the language, lyrics, and themes became highly localized. Parties gathered thousands of people in the poorest neighborhoods and mass media tried to describe the new phenomenon usually by discussing the relationship between the genre and hypersexualization, violence, and drug cartels \autocite{cunha_2013_tapatrao}.

In 2008, the genre migrated to the outskirts of São Paulo. The changes were not merely territorial. They were also comprised by substantive transformations in the underlying topics and symbols associated with the music. The celebration of violence was substituted by a brazen consumerist stand. The artists (also known as MCs like American rappers) show off designer clothes and imported cars and the lyrics have various references to highly expensive brands.

The praise of material goods and comforts christened the new style as \emph{funk ostentação} (in Portuguese, \enquote{ostentatious funk}). With no support of the big music recording companies, the MCs usually publicize their music videos through YouTube channels with remarkable success. Some videos have more than one million views \autocite{youtube_daleste, youtube_guime, youtube_charmes}. Some MCs perform four shows in the same night and earn around 10,000 Brazilian reais (US\$4,500) at each one of them \autocite{epoca_salario_guime}.

In most \emph{rolezinhos}, hundreds of poor youths gather in malls to flirt, sing the songs of their favorite MCs, and buy things. Like the leading artists of \emph{funk ostentação}, those young people also strive to wear and use products that are beyond their purchasing power. One of the most famous songs by MC Guimé sums up the spirit of those gatherings: \enquote{You are worth what you have.} A survey \autocite{fbiz_rolezinho_consumo} conducted by a Brazilian advertising agency listed some coveted brands for the participants in \emph{rolezinhos}: Victoria's Secret, Apple, Lacoste, Quiksilver, among others. Interviews performed by the same agency showed that many young people do not think twice before spending their monthly income on just one purchase and often incur considerable debt for the months to come.

There has been at least one study \autocite{abdalla_2014_rolezinho} that uses the framework of consumer culture theory in order to understand how the MCs build their identity based on brands, places, and objects associated with intensive consumption and, at the same time, spread that same identity among their young fans with remarkable effectiveness through videos and shows. In this context, it is easy to understand why the \emph{rolezinhos} unfolded in shopping malls. Not only are they some of São Paulo’s only public spaces, but also they are the spaces that capture the imagination and aspiration of those youths. 

It is also important to situate the \emph{rolezinho} phenomenon in the context of Brazilian democracy. After the military dictatorship in the 70s and the economic disaster of the 80s---with hyperinflation and stagnation---Brazil experienced two historical achievements.

In the mid-90s, inflation was defeated thanks to a successful economic plan outlined by then finance minister Fernando Henrique Cardoso, who became president in 1995 and ruled the country with his Brazilian Social Democratic Party (in Portuguese, \emph{Partido da Social Democracia Brasileira} or PSDB) for eight years \autocite{TWEC:TWEC472}.

In 2004, Workers' Party (in Portuguese, \emph{Partido dos Trabalhadores} or PT) candidate Luiz Inácio Lula da Silva won national elections and bolstered social welfare programs---especially the renowned \emph{bolsa família} (family allowance), that provides financial aid to poor families through direct cash transfers. In conjunction with a surge in the labor market and policies to increase the minimum wage, those social welfare programs halved the population living below the poverty line: from 22.6\% of the overall population in 2003 to 10.1\% in 2011 \autocite{rocha2012pobreza}.

Those youths in the \emph{rolezinhos} are the children of the stabilization of the 90s and the war on poverty in the early \nth{21} Century. Against the backdrop of such socioeconomic achievements, their consumerist fascination is easier to understand.

In June 2013, Brazil enacted its own \enquote{Spring.} It started with midsize demonstrations in São Paulo against a 10-cent increase in bus fares. This period saw the biggest demonstrations since the impeachment of Brazilian president Fernando Collor de Mello in 1992, after a corruption scandal. In São Paulo, protesters numbered 60,000. In Rio, around 100,000. The protesters' political spectrum was varied and their claims often antagonistic. The demonstrations brought together people who advocate reducing bus fares, the nationalization of the public transport service, the impeachment of the governor of São Paulo, broad political reform, boycotting the World Cup, more resources for education, among other issues.

For many social commentators, the June demonstrations remain a puzzle. In the streets, there were contradictory demands made by very distinct groups, the common denominator being a pervasive dissatisfaction with government corruption and the poor quality of public services. In general, the government had a awkward reaction in all levels. Both PT and PSDB---the two hegemonic parties---seemed totally lost in the face of unusual social upheaval \autocite{campos2013drives}.

As in the \emph{rolezinhos}, social networks helped to organize the June gatherings that started in São Paulo and then spread to other state capitals and major cities. The media made attempts to connect the June protests and the \emph{rolezinhos}, but June gatherings were clearly political \autocite{gohn2014social}, even if diverse and contradictory aims were represented.

The national and state elections this year further compounded to officials' worries about the possible negative impact of \emph{rolezinhos} on public opinion. At the same time, many of them did not refrain from attempting to use the debate to blame political adversaries for supposedly negligence or discrimination. It is worth remembering that both Brazilian president Dilma Rousseff and São Paulo mayor Fernando Haddad belong to the PT. On the other hand, São Paulo governor Geraldo Alckmin is an incumbent from the competitor PSDB. Therefore, the \emph{rolezinho} issue was at the crossroads of the Brazilian political struggle.

This is not a complete explanation of the \emph{rolezinhos} phenomenon, but should provide enough background for us to consider the question at hand: how this issue was framed and understood in Brazil's digital public sphere, which I will address in \autoref{chap:framing}.